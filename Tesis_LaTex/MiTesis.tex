%%%%%%%%%%%%%%%%%%%%%
% Documento maestro %
%%%%%%%%%%%%%%%%%%%%%
\documentclass{uanl}

%%%%%%%%%%%%%%%%%%%%%%%%%%%%%%%%%%%%%%%%%%%
% Cargando paquetes y definiendo opciones %
%%%%%%%%%%%%%%%%%%%%%%%%%%%%%%%%%%%%%%%%%%%
% Aquí puedes cargar los paquetes que vas a usar. La clase
% fime ya incluye babel, inputenc, graphicx y los de la AMS.
% Cargar un paquete está a tu libertad (y responsabilidad).
\usepackage[table,xcdraw]{xcolor}
\usepackage{hyperref}
    \hypersetup{breaklinks=true,colorlinks=true,linkcolor=black,citecolor=black,urlcolor=black}

% Ajusta la altura del número de página
\setlength{\footskip}{.8cm}

%%%%%%%%%%%%%%%%%%%%%
% Definiendo campos % 
%%%%%%%%%%%%%%%%%%%%%
\def\titulo{Título de la tesis}
\def\autor{Juan Jesús Torres Solano}
\def\matricula{2173262}
\def\grado{Maestría en Ciencia de Datos}
% En caso de que el grado tenga orientación o especialidad llenar el siguiente
% campo dejando un ESPACIO INICIAL, en caso contrario, dejar vacío
%\def\orientacion{Con Orientación en Fibras Ópticas, Fotónica y Sensores Ópticos.}
% Coloca el mes con mayúscula inicial
\def\fecha{marzo 2025}

\def\asesor{M.C. José Anastacio Hernández Saldaña}
\def\revisorA{Nombre del revisor A}
\def\revisorB{Nombre del revisor B}
% En el caso de que tu tesis sea de doctorado activa la variable cambiándola a \doctoradotrue
% y define tus otros dos revisores
\newif\ifdoctorado\doctoradotrue
\def\revisorC{Nombre del revisor C}
\def\revisorD{Nombre del revisor D}
% El visto bueno siempre va
\def\vobo{Dra. Azucena Yoloxóchitl Ríos Mercado}

%%%%%%%%%%%%%%%%%%%%%%%
% Inicia el documento %
%%%%%%%%%%%%%%%%%%%%%%%
\begin{document}

\frontmatter
\pagestyle{main}

\include{Capitulos/Portadas}
% Dedicatoria

\thispagestyle{empty}
\vspace*{17mm}

\begin{flushright}
\begin{itshape}

Aquí puedes poner tu dedicatoria\\
si es que tienes una.\bigskip\bigskip

\end{itshape}
\end{flushright}



\tableofcontents
\listoffigures
\listoftables

%Agradecimientos

\chapter{Agradecimientos}
\markboth{Agradecimientos}{}

Aquí puedes poner tus agradecimientos. (No olvides agradecer a tu comité de tesis, a tus profesores, a la facultad ).

%\include{Capitulos/Resumen}

\mainmatter
\pagestyle{uanl}

%%% Haz un documento para cada capítulo
\chapter{Introducción}

\section{Antecedentes o historia del arte}

\section{Justificación o motivación}

\section{Hipótesis}
\subsection{Hipótesis nula}
\subsection{Hipótesis alterna}

\section{Objetivos}

\subsection{Generales}

\subsection{Específicos}

\section{Metodología}




\begin{table}[ht]
	\centering
	\caption{Una tabla}
	\label{tab:una-tablita}
		\begin{tabular}{|c|c|}
			\hline
			a & b \\
			\hline
			c & d \\
			\hline
		\end{tabular}
\end{table}



\begin{figure}[htp]
	\centering
		\includegraphics[width=14cm]{graf}
	\caption{gráfica 1}
	\label{fig:uanl}
\end{figure}

\subsection{Matemáticas}



La ecuación \eqref{eq:obvia} es obvia.
\begin{equation}
	x + x = 2x
	\label{eq:obvia}
\end{equation}



\begin{definicion}
	Un número entero es primo si y sólo si tiene exactamente cuatro divisores distintos.
\end{definicion}



\begin{teorema}[Fermat]
	No existen $x,y,z \in \mathbb{Z}$ (que no sean los triviales) tales que $x^n + y^n = z^n$ para $n \geq 3$.
\end{teorema}

\chapter{DELIMITACIÓN Y PLANTEAMIENTO DEL PROBLEMA}

Durante un trabajo de prospección geofísica, al realizar adquisición de datos geoeléctricos \textit{in situ}, no es posible conocer el resultado del trabajo hasta una vez realizado procesos de inversión de datos resultado de prospección, por lo que no se tiene certeza de si una lectura presenta inconsistencias o si forma parte de una respuesta esperada para determinado resultado.\\

La problemática radica en que el muestreo planeado, en la etapa de planeación de adquisición, puede no ser efectivo o no reflejar una distribución esperada, debido a que el medio que se prospecta carece de homogeneidad, siendo solo evidente durante la exploración directa del medio, lo que en muchos casos no se logra a identificar, sumando ambigüedad al proceso de interpretación.\\ 

Como primera aproximación a una solución se considerara un primer supuesto, en el cual se cuenta con respuestas de sondeos eléctricos verticales, más simple pero no menos complejo de interpretar, al contar con un solo segmento de señal, dentro de esta señal podemos encontrar distintas unidades geológicas, profundidad de acuífero, espesores de unidades y profundidad de exploración.\\

La propuesta de solución es mediante Aprendizaje automático (ML, por sus siglas en inglés), empleando una técnica de aprendizaje que permita identificar patrones que estén asociados a la respuesta de un modelo de inversión, de esta manera obtener un modelo pronóstico de la inversión, sirviendo de guía para incrementar la densidad de muestreo para mejorar el modelo y respuestas.

\chapter{JUSTIFICACIÓN}

Existen múltiples aplicaciones de ML en el procesamiento, modelado e interpretación geofísica \citep{li2024, liu2020,el2001,wrona2018}, algunas de estas aplicaciones corresponden a implementaciones académicas, al igual que comerciales, mayormente desarrolladas en software de exploración sísmica de hidrocarburos \citep{diaferia2024high, panebianco2024automated}.


%Es de esperar que al implementar ciencias de datos en diversas áreas de la ciencia, se optimicen aspectos básicos, mejorando la analítica e interpretación de resultados, o bien, modelando predicciones de tendencias.

La aplicación de ML durante la ejecutar de muestreo de Sondeo Eléctrico Vertical (SEV), en particular durante el proceso de adquisición \textit{in situ}, presenta la posibilidad de evaluar y ampliar el intervalo de muestreo original, mejorando la adquisición tradicional, realizando una predicción de muestreo en intervalos intermedios o finales durante la adquisición, identificando nuevos intervalos de apertura de electrodos no considerados previamente.

Durante el muestreo tradicional los intervalos se diseñan considerando el objetivo de exploración (idealmente), este análisis previo es un factor determinante, en esta etapa permitiendo establecer el intervalo de muestreo que permitirá identificar el objeto de exploración, o anomalía, su respuesta puede asociarse a unidades geológicas, acuíferos, fallas, zonas de fracturas, estructuras antropogénicas, infraestructura moderna, etc.

De manera general se busca mantener un intervalo de muestreo menor a la frecuencia de ocurrencia del objetivo de estudio, por lo que el éxito de la exploración dependerá en su totalidad de la planeación de la adquisición, lo que implica conocer previamente la conformación, distribución y espesor de cada unidad, aun contando con esta información, puede resultar complicado el procesado e  interpretación de las anomalías, por la misma ambigüedad del ajuste de los modelos que satisfagan las curvas de inversión y la propia heterogeneidad del medio y de sus propiedades.

De manera que un modelo entrenado permita generar múltiples predicciones, en intervalos no explorados, e identificar regiones no cubiertas por el muestreos, permita contemplar la mejora de la adquisición con puntos adicionales \textit{in situ} mejorando el intervalo de lectura e impactando en la calidad del muestreo, ajustando la respuesta geoeléctrica.

Pese a existir aplicaciones de ML implementadas en geofísica, no se identifica alguna enfocada este esta problemática en concreto (citar estudios de australia), %sin embargo, hay ejemplos en otros campos de estudio con un enfoque en el muestreo y clasificación de datos no paramétricos \citep{entezami2022non, bkassiny2013multidimensional, shi2021non} empleando técnicas como Dirichlet Process Mixture Model (DPMM), radial basis function network (RBFN),  Multiple Point Statistics (MPS) y Bayesian Compressive Sensing (BCS).

Para poder abordar la problemática se requiere de un modelo robusto ante el ruido, que permita trabajar con datos no paramétricos, sea favorable a la distribución de los datos, pueda establecer un modelo mediante entrenamiento supervisado, con capacidad de ejecutar regresiones a partir de entrenamientos previos y permita realizar pronósticos de valores de resistividad. Estas condiciones son cubiertas por tres modelos de ML Random Forests (RF), Support Vector Machines (SVM) y Gradient Boosting Regression (GBR), de los cuales mediante su evaluación, por medio de la puntuación de predicción, la facilidad de implementación y ejecución en pruebas con datos reales,  permita identificar el que presenta el mejor rendimiento y ajuste para su implementación en la adquisición geofísica. 

\chapter{OBJETIVO}



\section{Generales}



\section{Específicos}


\chapter{MARCO TEÓRICO}
	\section{Geofísica y Geoeléctrica}
		\subsection{Definición de Geofísica}
			
			En términos generales la geofísica es la aplicación de los principios físicos de la materia en el estudio del planeta Tierra, o cual quier otro cuerpo celeste, desde el campo magnético, pasando por los fenómenos atmosféricos al medio solido del subsuelo, hasta las profundidades del núcleo interno planetario, ya sea que se empleé una fuente natural como la propagación de ondas elásticas generadas por sismicidad, ó bien, la inducción de campo electromagnético de fuente controlada \citep{parasnis2012, reynolds2011, lay1995}.   
		
			El  nacimiento de la geofísica es relativamente reciente, la primera prospección geoeléctrica data de 1830 realizados por \cite{fox1830} en Cornwal, Reino Unido, donde aplico técnicas de Self-Potential en exploración de mineralización de sulfuro en vetas, la medición del potencial natural resulto altamente efectiva para la prospección de este tipo de mineralizaciones ya que su anomalía se caracterizaba por presentar una respuesta muy marcada con respecto al medio \citep{reynolds2011, revil2013}.
			
		\subsection{Resistividad de la Tierra}
				
			De manera general la materia presenta características definidas a partir de los elementos que la integran, en primer orden la configuración atómica establece las propiedades físicas corresponden a la estructura de electrones, protones y neutrones que presentan los átomos; a su vez, las moléculas pueden estar conformadas por una clase especifica de átomos (moléculas homonucleares) o por conjuntos de diferentes tipos (compuestos), cuya conformación depende de factores físico-químicos \citep{tiab2024}.
			
			La configuración molecular inorgánica presente en la materia, definirá el tipo de estructura cristalina (mineral) que formarán, en conjunto; esta configuración cristalina es la que encontramos en el medio geológico conformando los minerales que componen la estructura mineral de una unidad geológica (ver figura \ref{fig:eo}) \citep{gandhi2016, tiab2024}.\\
			
			\begin{figure}[h!]
				\centering
				\includegraphics[width=9cm]{Imagenes/estructura-oro}
				\caption[Estructura atómica del oro]{Esquema de la estructura atómica de oro que conforma la cristalización octahedral, modificado de \citet{sorrell1973}}.
				\label{fig:eo}
			\end{figure}
			
			Los métodos Geoeléctricos se clasifican en dos grupos, métodos pasivos y de inducción, los primeros corresponden a aquellos en los que se mide el potencial eléctrico natural, usualmente medido en mili volts, en donde se requiere de electrodos no polarizables para tener medidas lo más claras posibles; mientras que los métodos de inducción emplean un arreglo de electrodos, o inductores de campo electromagnéticos, mediante los cuales se induce un campo eléctrico al subsuelo, calculando la diferencia de potencia eléctrica en el medio, o bien, el decaimiento de la polarización inducida \citep{revil2013, reynolds2011, igboama2023}.
			
			Los métodos de inducción, Sondeo Eléctrico Verticales (VES, por sus siglas en inglés), Tomografía de Resistividad Eléctrica (ERT, por sus siglas en inglés), Polarización Inducida (IP, por sus siglas en inglés), presentan una gran ventaja ya que no dependen del medio para poder realizar una lectura, ademas de poder realizarlos en cualquier momento, manteniendo el equipo en condiciones de operación, y pode diseñar arreglos de adquisición que nos permitan tener un muestreo tan amplio o limitado como sea conveniente, solo limitados por el alcance y potencia de los equipos empleados. Por otro lado su interpretación presenta un alta ambigüedad, solo acotado por la cantidad de referencias que puedan cruzarse para robustecer el modelo geológico y de inversión, y así poder llegar a una interpretación satisfactoria \citep{reynolds2011, igboama2023}.
			
			El método de prospección geoeléctrica, en especifico el SEV y la TRE, consiste en determinar la distribución de resistividades del subsuelo, de manera que se pueda establecer una correlación entre la resistividad y un modelo ajustado a la realidad geológica-estructural, geotécnica o geohidrológica del objeto de estudio.
	
		\subsection{Sondeo Eléctrico Vertical}
			
			Los SEV corresponden al método de mas rápida ejecución y económicamente mas accesible, por lo que es ampliamente empleado para solucionar problemas de ingeniería, minería, geotecnia, monitoreo e impacto ambiental y abastecimiento de Aguas potable; siendo de gran utilidad en la exploración de hidrogelogica ya que la respuesta resistiva de un medio saturado permite establecer diferencias concisas y discriminar entre agua dulce, salada, rocas fracturadas, arcillas , arenas, conglomerados, etc.
			
			La resistividad es medida mediante la inyección de una corriente en el subsuelo y mientras que se monitorea y captura la diferencia de potencial eléctrico en la superficie, esta lectura corresponde al valor de la contribución resistiva de todas las capas por donde fluye la corriente.
			
			La inyección de corriente y medición del potencial se realiza a través de un arreglo de dos pares de electrodos, $A, B (C_{1}, C_{2})$ y $M, N (P_{1}, P_{2}) $ respectivamente, siendo el electrodo $A (C_{1})$ el polo positivo y $B (C_{2})$ el polo negativo de inyección, mientras que el electrodo $M (P_{1})$ corresponde al polo positivo y $N (P_{2})$ al polo negativo de los electrodos de potencial.\\
			 
			\begin{figure}[h!]
				\centering
				\includegraphics[width=9cm]{Imagenes/ArregloElectrodos}
				\caption[Configuración general de electrodos]{Configuración general de arreglo de electrodos, modificado de \citet{reynolds2011}}.
				\label{fig:AE}
			\end{figure}
			La resistividad del subsuelo se calcula a partir de la ley de Ohm, considerando el caso general en donde el medios es homogéneo y el arreglo de electrodos presenta una distribución convencional, donde se establece una relación directamente proporcional entre la la resistencia $R$ ,medida en Ohm ($\Omega$), y el cociente entre la diferencia de potencial $\Delta V$ y la corriente inducida $I$, para un valor puntual \citep{igboama2023}.
			
			\begin{equation}
				R = \frac{\Delta V}{I}
			\end{equation}
			   
			Sabiendo que se puede calcular $R$ para una sección con longitud $L$ y un área $A$, transversal del material, conociendo la resistividad ($\rho$) del material \citep{igboama2023, lowrie2020}, podemos reescribir la ecuación como: 
			
			\begin{equation}
				R = \rho \frac{L}{A}  \rightarrow  	\rho  = R \frac{A}{L} \rightarrow \rho  = R \cdot k
			\end{equation}
			
			Donde la resistividad ($\rho$) es una constante de proporcionalidad del medio y $k$ es el factor geométrico de distribución del flujo de corriente en términos de la del arreglo  de los electrodos de inducción y potencial (distancias entre los electrodos A-M-N-B ) \citep{igboama2023, lowrie2020}.
			
			\begin{equation}
				k = 2\pi \left(  \dfrac{1}{AM} - \dfrac{1}{AN} - \dfrac{1}{BM} + \dfrac{1}{BN} \right) 
			\end{equation}
			
			Tenemos que la resistividad aparente ($\rho _{A}$) de una sección del subsuelo, corresponde a la contribución resistiva de las unidades geológicas en esa sección, en términos de las distancias entre electrodos, la diferencia de potencial y el flujo de corriente en el medio \citep{igboama2023, lowrie2020}, esta dado por la siguiente ecuación:
			
			\begin{equation}
				\rho_{A} = \frac{\Delta U}{I} \cdot k %\rho_{A} = 2\pi \cdot \frac{\Delta U}{I} \cdot k
			\end{equation}
			
			\subsubsection{Arreglo de Electrodos y Factor Geométrico}
			
				Cada arreglo presenta ventas, desventajas, rango de sensibilidad y espacio de ejecución, debido a estas características y se tiene que evaluar e identificar que arreglo cumple con las condiciones adecuadas para ser ejecutado, considerando el espacio disponible en el sitio de estudio, el nivel de ruido (motores, conexiones a tierra mal aterrizadas, antenas, postes metálicos, arboles), la profundidad de objeto de prospección y la resolución vertical alcanzable (ver figura \ref{fig:Contri}).
				
					\begin{figure}[h!]
						\centering
						\includegraphics[width=12cm]{Imagenes/5}
						\caption[Esquema de la contribución de la respuesta eléctrica]{Esquema de la contribución de la respuesta de resistividad eléctrica, modificado de \citet{reynolds2011}}.
						\label{fig:Contri}
					\end{figure}				
				
				Como se observa en la sección anterior, la resistividad se determina empleando una configuración de los electrodos durante una medición, las distintas configuraciones de electrodos se encuentran ampliamente documentadas, cada una presenta un factor geométrico distinto \citep{igboama2023, lowrie2020}, los principales arreglos geoeléctricos son:
				
				\begin{description}
					\item[Wenner ]  
							\begin{equation}
								\rho_{A} = 2\pi \cdot R \cdot a
							\end{equation}
							
							\begin{figure}[h!]
								\centering
								\includegraphics[width=9cm]{Imagenes/1}
								\caption[Esquema del arreglo Wenner]{Esquema del arreglo Wenner, modificado de \citet{reynolds2011}}.
								\label{fig:AW}
							\end{figure}
							
					\item[Schlumberger ] 
					
						\begin{equation}
							\rho_{A} = \frac{\pi a^{2}}{b} \left[ 1 - \frac{b^{2}}{4 a^{2}} \right] \cdot R, \quad a \geq 5b
						\end{equation}

							\begin{figure}[h!]
								\centering
								\includegraphics[width=9cm]{Imagenes/2}
								\caption[Esquema del arreglo Schlumberger]{Esquema del arreglo Schlumberger, modificado de \citet{reynolds2011}}.
								\label{fig:AS}
							\end{figure}
					
					\item[Dipolo-dipolo]  
					
							\begin{equation}
								\rho_{A} = \pi n(n+1)(n+2)a \cdot R
							\end{equation}
					
							\begin{figure}[h!]
								\centering
								\includegraphics[width=9cm]{Imagenes/3}
								\caption[Esquema del arreglo Dipolo-dipolo]{Esquema del arreglo Dipolo-dipolo, modificado de \citet{reynolds2011}}.
								\label{fig:ADD}
							\end{figure}
				\end{description}
	
	\section{Adquisición de Datos Geofísicos}
	
	Previo al trabajo de adquisición se realiza un análisis de entorno, en el cual se verifica la viabilidad del arreglo dadas las condiciones del sitio, considerando lo siguiente: espacio disponible en el sitio de estudio, profundidad de exploración, nivel de ruido eléctrico, interferencias con la estabilidad del potencial natural del subsuelo, profundidad del objeto de exploración y dimensiones aproximadas del mismo.
	
		\subsection{Intervalo de Muestreo en SEV}
			
			El intervalo de muestreo empleado durante la adquisición de un SEV es un parámetro crítico que influye en la calidad y precisión de los datos geofísicos adquiridos, ya que esta estrechamente relacionado con la resolución vertical que deseamos de acuerdo al objeto de estudio. Durante la planeación es necesario considerar distintas condiciones, como son:
			
			\begin{itemize}
				\item Los espesores de cada unidad.
				\item La distribución de las distintas unidades.
				\item Profundidad de investigación
				\item Ruido en la señal.
			\end{itemize}
			
			Para establecer un intervalo de muestreo apropiado, se deben considerar el Teorema de Muestreo de Nyquist y El teorema de Shannon-Hartley (teorema de codificación de canal ruidoso)
			
			El Teorema de Muestreo de Nyquist, el cual, es un principio fundamental en el procesamiento de señales analógicas y digitales, donde establece las condiciones mínimas necesarias para una reconstrucción una señal analógica a partir de muestras discretas \citep{alvarado2010}.
			
			El teorema de muestreo de Nyquist nos garantiza las condiciones necesarias y suficientes para llevar a cabo una adquisición exitosa de muestreo de una señal, llámese distribución de resistividad en un medio heterogéneo y discontinuo \citep{alvarado2010}.
			
			\begin{equation}
				f_{s} \geq 2 \cdot f_{max}
			\end{equation}
			
			Donde la frecuencia de muestreo $f_{s}$ es por lo menos dos veces mayor a la frecuencia máxima $f_{max}$ conocida, cuando el teorema no se cumple se genera una distorsión en la señal, sumando las frecuencias altas incompletas a la señal natural de baja frecuencia, generando ruido, y problemas de interpretación, se conoce como aliasing \citep{alvarado2010}. 
			
			Considerando el medio geológico como una región con presencia coanstante de ruido electrico de fuentes tanto naturales como humanas, es impresindible considerar el teorema de Shannon-Hartley aplicando apilamiento de muestreo como metodo de reduccion de la relacion ruido señal, durante la adquisicion de datos; esto quiere decir calcular el promedio de muestreos cointinuos en un intervalo definido de aperturas entre electrodos.
			
			%% agregar ecuacion de shanon y desgrlosar sus elementos e interpretacion
				
			\subsubsection{Factores que Determinan el Intervalo de Muestreo}
		
				En el contexto de la adquisición de datos mediante SEV, el intervalo de muestreo es equivalente al espaciado entre puntos donde se realizan mediciones de resistividad del subsuelo. Este intervalo de muestreo debe ser lo mas pequeño posible, de modo que permita obtener muestras de resistividad \citep{telford1990}, esta relación se define de la siguiente manera:
				
				\begin{equation}
					f_{s}= \frac{1}{\Delta x}
				\end{equation}
				
				donde el intervalo de muestreo $\Delta x$ debe ser menor a la mitad de la longitud de onda ($\lambda_{min}$, espesor) asociado al objetivo de exploración
				
				\begin{equation}
					\Delta x \leq \frac{\lambda_{min} }{2}
				\end{equation}
				
				
			
		\subsection{Proceso de Adquisición In Situ}
			La adquisición de datos se realiza mediante la lectura directa en campo, al inducir corriente continua empleando un resistivimetro mediante de los electrodos de corriente A ($C_{1}$) y B ($C_{2}$), mientras se realiza la lectura de potencia en los electrodos M ($P_{1}$) y N ($P_{2}$), la lectura se realiza en intervalos regulares en instantes de inyección de corriente distintos \citep{telford1990}.
			
			Durante la toma de datos es importante considerar los modelos previos realizados durante el análisis preliminar, ya que las resistividades esperadas para las unidades, permiten tener control en la dispersión de datos, identificando tomas erróneas y corrigiendo al momento con una nueva lectura \citep{telford1990}.
	
	\section{Machine Learning (ML) en la Geofísica}  
	
	La aplicación de ML y el DL en la geofísica es ampliamente utilizado en exploración sísmica, abarcando los procesos de adquisición, procesado e interpretación, mejorando los tiempos de procesamiento, clasificación e interpretación, ya que es en este método donde se cuenta con la mayor cantidad de datos para entrenamiento \citep{wrona2018}; en menor medida se implementan técnicas de ML en la exploración y prospección geoeléctrica, hay algunos ejemplos destacables como son \cite{liu2020, el2001, li2024}, sin embargo no es un estándar en la industria, pese a las ventajas que puede tener su aplicación, como es el caso de este estudio
		
	El aprendizaje automático o machine learning, son un conjunto de técnicas que utilizan algoritmos con los cuales permite a un sistema aprender y generar predicciones, para lo que requiere un conjunto de datos para poder realizar el entrenamiento. Podemos clasificar los algoritmos de ML de dos maneras, por el tipo de aprendizaje, correspondiendo a Aprendizaje supervisado, no supervisado y por refuerzo, y por la relación que establecen con los parámetros del conjunto de datos de entrenamiento, es decir, modelos paramétricos y no paramétricos \citep{li2024}. 
	
	De los modelos no paramétricos destacan por su adaptabilidad a la estructura subyacente de los datos, por lo que pueden realizar aprendizaje de relaciones complejas entre datos, así como ausentes de linealidad, teniendo un costo en volumen de datos, requiriendo un número mayor para su entrenamiento, destacan los algoritmos siguientes.
	
	\begin{itemize}
		\item Árboles de decisión
		\item Random Forests
		\item K-Nearest Neighbors (KNN)
		\item Máquinas de soporte vectorial (kernelizados)
	\end{itemize}
	
	Dada la naturaleza de los datos de SEV's, heterogéneos, discontinuos y no lineales, es conveniente abordar su analistas desde un enfoque no paramétrico, teniendo esto en cuanta, la técnica Random Forests destaca siendo eficaz en la tarea de clasificación y regresión, teniendo algunos beneficios como son la reducción del sobre ajuste, interpretación de variables, resistencia al aliasing.
	
	\section{Random Forests} 
	
		La técnica Random Forests emplea múltiples arboles de decisión independientes entre si, donde cada árbol realiza una votación de clases, donde se selecciona la más popular de la entrada de cada árbol realizando una combinación de salida, permitiendo realizar una clasificación de características complejas o realizar regresiones de datos complejos multivariables \citep{breiman2001, lan2020}.
		
		La herramienta de Random Forests, de acuerdo con \citet{breiman2001} emplea tres elementos clave en el proceso de entrenamiento, bagging, selección aleatoria de características y agregación por votación, resultando en la combinación del los resultados en una predicción o clasificación robusta y ajustada \citep{lan2020}.
		
		\subsection{Siembra del bosque}
		
			\citet{breiman2001} nos dice que Random Forests es un conjunto de clasificadores $H(x,\theta_{k})$, $x$ es un vector de entrada y $\theta_{k}$ corresponden a vectores aleatorios independientes.
			
			A partir de los datos de entrada, se generan subconjuntos de datos de entrenamiento, estos se seleccionan con cierta aleatoriedad empleando la técnica bootstrap sampling, en cada nodo de los subconjuntos de entrenamiento se selecciona un subconjunto de características por votación de popularidad, dejando crecer cada árbol sin realizar poda hasta completar los criterios de finalización, es decir un numero de instancias preestablecido\citep{breiman2001}.
			
		\subsection{Predicciones del bosque}
		
			
			 La salida de un Random Forest para una entrada $x$ se basa en las predicciones individuales de los árboles para cada clase $h_{k}(x)$, se realiza un conteo de cada clase, producto de la predicción de cada árbol, sumando las salidas $I(h_{k}(x)=c)$, y finalmente se selecciona clase con mayor numero de predicciones, obteniendo la predicción de clasificación $H(x)$,donde $x$ es una función indicadora que vale 1 si $h_{k}(x)=c$, y $0$ en caso contrario \citep{breiman2001}.
			
			\begin{equation}
				H(x) = \text{argmax}_c \sum_{k=1}^K I(h_k(x) = c)
			\end{equation}

			
			El proceso de la regresión se obtiene a partir de la media aritmética de cada predicción individual, donde cada árbol produce un valor numérico $h_{k}(x)$ correspondiente a cada $x$, al corresponder con promedio de las predicciones se le otorga mas estabilidad cuando tenemos un numero elevado de arboles  y un conjunto de datos grande, entendiéndolo como un modelo central que incorpora información de cada árbol \citep{breiman2001}. 
			
			\begin{equation}
					H(x) = \frac{1}{K} \sum_{k=1}^K h_k(x)	
			\end{equation}

		\subsection{Margen y Error de Generalización}
		\subsection{Robustez y Convergencia}
		\subsection{Aplicaciones de Random Forests en Geofísica}
	
\chapter{METODOLOGÍA}

%\section{ideas y apuntes}


%\textbf{Adquisición de datos y preprocesamiento: En primer lugar, se realiza la adquisición de datos in situ utilizando el método de Sondeo Eléctrico Vertical (SEV), el cual consiste en medir la resistividad del terreno a diferentes profundidades con un intervalo de muestreo predefinido. Este intervalo se determina de acuerdo con el objetivo de exploración, como la identificación de unidades geológicas, acuíferos, fallas, fracturas o estructuras antropogénicas. Para garantizar que los datos sean adecuados para el análisis, se lleva a cabo un preprocesamiento de los datos, que incluye la limpieza de valores atípicos, normalización de las lecturas y manejo de valores faltantes.}

	El objetivo central de la metodología consiste en evaluar la eficacia de la generación de pronósticos de resistividad aparente a partir del entrenamiento supervisado de los modelos Random Forest, Gradient Boosting Regression y Vector Sopporter Machine.
	
	En ciencia de datos la cadena de procesos de análisis, se estructura y diseña rigurosamente para garantizar la calidad de los datos de entrada, la preparación y diseño de los datos de entrenamiento, la optimización de los modelos y la evaluación de los resultados, así como la comparación de cada modelo empleado.
	
	El proceso metodológico se dividió en 5 etapas:
	
	\begin{enumerate}
		\item Análisis de datos iniciales y simulación
		\item Limpieza y preparación de datos
		\item Modelado y entrenamiento
		\item Validación, optimización y comparación de modelos
		\item Test y evaluación
	\end{enumerate}
	

	\section{Análisis de datos iniciales y simulación}
	
	Durante las etapas de adquisición "in situ", se cuenta con información limitada generada en gabinete y los datos obtenidos durante la adquisición geofísica, estos últimos correspondiendo a profundidades aparentes de exploración, aperturas de electrodos de corriente y potencial, así como a los resultados de resistividad aparente resultantes de cada lectura.
	
	Ya que se cuenta con un numero limitado de variables, $\rho _{A}$, $AB/2$, $Z$ y el modelo de unidades geológicas esperado, así como su valor de resistividad inferido, estos datos conforman los elementos básicos necesarios durante la exploración y por consiguiente se integran como base para el entrenamiento de los modelos.
	
	Antes de poder ser empleadas las variables identificadas, es necesario establecer la fuentes de los datos, definir el formato adecuado y cubrir la dimensión y estructura necesaria en los datos para lograr un entrenamiento efectivo. 
	
		\subsection{Datos iniciales}
		
			Se cuenta con un total de 8 sitios de exploración, los cuales están integrados por al menos con 2 Sondeos Eléctricos Verticales (SEV), en diferentes ambientes geológicos, reuniendo un total de 26 SEV's, correspondientes a sitios de exploración minera, geohidrológica y geotécnica, los cuales se modelaron, interpretaron y validaron mediante exploración por perforación o excavación.
			
			Con el objetivo de reducir el sesgo, no se emplean los datos de resistividad aparente medido in situ, siendo empleados en el test, en su lugar se emplean los modelos validados de cada sitio y sondeo, integrados por datos de espesores de unidades geológicas y las resistividades resultantes del modelo de inversión para cada espesor.
			
			En este punto es importante considerar cuatro aspectos importantes, la heterogeneidad del subsuelo, las condiciones geotécnicas particulares del medio, el principio de equivalencia y la densidad de datos óptimos requeridos por los modelos para su entrenamiento; los dos primeros corresponden a factores cruciales en el comportamiento eléctrico del medio y su respuesta; los últimos dos atañe a la complejidad de interpretación, relacionada con la cantidad de modelos requeridos para el entrenamiento.
			
			Como estrategia, ante estos enormes obstáculos, se opto por una postura centrada en la aleatoriedad de las propiedades resultantes (espesor y resistividad), empleando los rangos respectivos de cada sitio, teniendo en cuanta que los resultados obtenidos corresponden a condiciones geológicas similares, donde se observan mismas unidades, pero con distinta respuesta geoeléctricas.
			
			Al considerar un estado "aleatorio" entre los rangos de valores de las variables evitamos ajustarnos a una distribución pre establecida, evitando así overfitting inicial, lo cual permite entrenar los modelos con todas las variaciones posibles.
			
			 
			
			
			
			
			
			
			
			Para el entrenamiento se emple 
			
			
			
			\subsubsection{Limpieza de datos}
			
		
		
	
		


	\section{Limpieza y preparación de datos}

	\section{Modelado y entrenamiento}

	\section{Validación, optimización y comparación de modelos}
	
	\section{Test y evaluación}
	
\include{Capitulos/Capitulo7}

\appendix
%%% Haz un documento para cada apéndice, si es que tienes
\chapter{Este es un apéndice}






\backmatter
\pagestyle{main}

% Aquí va la bibliografía, puedes usar el entorno de LaTeX (thebibliography)
% o la herramienta BibTeX. En caso de que optes por BibTeX, puedes usar
% alguno de los archivos de estilo (mighelbib.bst o mighelnat.bst) incluidos
% en el paquete, cuyos diseños armonizan con el diseño de tesis provisto por
% fime.cls. Para muestra, basta un botón:
\bibliographystyle{mighelnat}
\bibliography{MiBiblio}

\label{lastpage}
%\include{Capitulos/Autobiografia}

\end{document}

