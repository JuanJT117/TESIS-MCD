\chapter{FORMULACIÓN DE OBJETIVOS}

\section{Generales}

\begin{itemize}
	\item Generar un modelo predictivo que permita clasificar datos geoeléctricos durante la adquisición de SEV's \textit{in situ}.
	\item Establecer un modelo de entrenamiento de regresión mediante Random Forest que genere intervalos de muestreo en SEV's al identificar contrastes en los datos.
\end{itemize}

\section{Específicos}


\begin{itemize}
	\item Establecer modelos de geoeléctricos con unidades geológicas de terrenos ya caracterizados mente SEV's.
	\item Generar un modelo de clasificación y regresión mediante Random Forest mediante modelos geoeléctricos.
\end{itemize}
\newpage
\chapter{MARCO TEÓRICO}
	\section{Método Geoeléctrico}
		La geofísica es una rama de la ciencia relativamente reciente, la primera prospección geoeléctrica data de 1830 realizados por \cite{fox1830} en Cornwal, Reino Unido, donde aplico técnicas de Self-Potential en exploración de mineralización de sulfuro en vetas, la medición del potencial natural resulto altamente efectiva para la prospección de este tipo de mineralizaciones ya que su anomalía se caracterizaba por presentar una respuesta muy marcada con respecto al medio (\cite{revil2013}; \cite{reynolds2011}).\\
		
		Los métodos Geoeléctricos se clasifican en dos grupos, métodos pasivos y de inducción, los primeros corresponden a aquellos en los que se mide el potencial eléctrico natural, usualmente medido en mili volts, en donde se requiere de electrodos no polarizables para tener medidas lo mas claras posibles; Mientras que los métodos de inducción se emplea un arreglo de electrodos, o inductores de campo electromagnéticos, mediante los cuales se induce un campo eléctrico el subsuelo, calculando al diferencia de potencia eléctrico en el medio, o bien el decaimiento de la polarización inducida en el medio.\\
		
		
		\subsection{Principios Básicos}
		
			explicación teórica de la física y matemática del método geoeléctricos
			
		\subsection{Configuración de electrodos}
			%%%%    sin contenido
			\subsubsection{Caso General}
				%%%%	explicacion del metodo en el caso particular de un medio homogenio
				Si consideramos un medio continuo y homogéneo explicacion del metodo mediante el caso particular.\\
				  
			\subsubsection{Arreglos Convencionales}
				Metodo Wenner, método Shlumberger, dipolo dipolo, polo dipolo, ecuatorial.\\
				
		\subsection{Sondeo Eléctrico Vertical}
		caso particular de sondeos, distribucion de señal, interpretacion, caso particular de exploracion hidrologica, señales caracteristicas de acuiferos, modelos reales y simulaciones.\\
		
	\section{Árboles de Decisión}
	\section{Random Forest}
		\subsection{Número de Árboles}
			\subsubsection{Max Tree Depth}
		\subsection{Bootstrap}
		\subsection{Overfitting y Underfitting}
		\subsection{OOB Score}
		\section{Validación Cruzada}
		\section{Métricas de Evaluación}
		
			
		