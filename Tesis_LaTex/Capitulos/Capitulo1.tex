\chapter{INTRODUCCIÓN}
Durante las etapas iniciles de proyectos de ingenieria civil, vial, petrolera, geohidrologica o minera, se requiere conocer el medio sobre el cual se desarrollaran los proyectos, ya sea para conocer el estado inicial del medio que sera intervenido o del cual se busca algun beneficio o remediacion que solo sea interpretable mediante exploracion indirecta.

Bajo estas circunstancias se emplean metodos indirectos no destructivos para explorar el subsuelo, resultando en un amanera eficiente y rapida para conocer las caracteristicas del medio.

Una de estas metodologias corresponde a la prospeccion geoelectrica, la cual aprobecha las caracteristicas de conductividad del subsuelo para medir la resistencia la paso del flujo electrico atrabas de las distintas capas del subsuelo, permitiendo conocer la contribucion de la resitividad para una seccion del medio.

Esta informacion es generada in situ emplenado ditintas configuraciones de lectrodos de lectura y de inyecciond e corriente, y es en la configuracion de estos electrodos donde a partir de la las ecuaciones de flujo elecrico se describe la forma en que este se dispersa en el medio, permitiendo donocer la profundiad aparente de exploracion y la resitividad aparente del area del flujo electrico.

Es aqui donde se pretende establecer una relacion de predictibilidad entre un medio conocido y nuevos levantamientos
 


oportunidades de la cuiencia de datos en el análisis en tiempo real de la respuesta 



