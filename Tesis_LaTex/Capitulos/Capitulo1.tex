\chapter{INTRODUCCIÓN}

La aplicación de ciencia de datos en geofísica presenta oportunidades para su ejecución, siendo mayormente desarrollada en sísmica petrolera, considerando y aprovechando la gran cantidad de datos que se generan en las campañas de exploración, así como los datos interpretados de las mismas, otorgando elementos cruciales para el entrenamiento de modelos que permitan el reconocimiento de patrones en los datos, así como su etiquetado y correlación con secuencias geológicas; representando una oportunidad vigente de aplicación en métodos geofísicos. 

La geofísica se encarga de explorar el medio terrestre, empleando métodos que aprovechan las propiedades físicas del subsuelo como medio de respuesta ante la interacción activa o pasiva, ya sea mediante a la inducción de una campo eléctrico o mediciones de las variaciones magnéticas de la tierra respectivamente, entre otros métodos, permite modelar e interpretar la respuesta del medio de acuerdo al objetivo de estudio.  

El método geofísico de prospección geoeléctrica aprovecha las propiedad de resistividad para caracterizar las distintas unidades del medio geológico, a partir de su capacidad de resistividad eléctrica, siendo de interés para este trabajo el método de inducción eléctrica en su modalidad de Sondeo Eléctrico Vertical (SEV).

Los SEV se realizan siguiendo un arreglo geométrico de 4 electrodos previamente definido (ver figura \ref{fig:AE}), el cual puedo ejecutarse siguiendo la configuración Wenner, Dipolo-Dipolo o Schlumberger, entre otros, cada arreglo presenta un patrón distinto de dispersión del flujo eléctrico, cambiando su sensibilidad ante variaciones verticales u horizontales. 

Durante la adquisición de datos en campo la apertura entre los electrodos cambia en razón al arreglo geoeléctricos empleado, de manera que obtenemos para cada cambio en apertura un valor de resistividad aparente, $Rha$ o $\rho_{a}$ en $(\varOmega\ m)$, posterior al proceso de modelado e inversión, obtenemos una distribución de espesores con una resistividad asociada a cada uno de ellos.

Convencionalmente la prospección eléctrica se aplica, procesa e interpreta siguiendo una metodología claramente establecida, esto no la exime de ser viable la mejora de cada uno de sus aspectos.

Algunas etapas de este proceso an sido optimizadas, como la automatización de la lectura de datos, dicho esto, en el proceso de planeación de adquisición y validación de lecturas en campo pueden mejorar por medio de modelos de aprendizaje, los cuales requieren un amplio numero de datos para su entrenamiento.

Para realizar un entrenamiento efectivo, se requiere de un gran numero de datos, este problema se solventa mediante la simulación de variaciones de modelos de sondeos interpretados y verificados por exploración directa, siendo esto plausible al considerar la heterogeneidad del medio y manteniendo las variaciones dentro de los rangos de valor identificados en un mismo sitio.

Los datos de simulación \textit{($\rho$, $H$)} se conforman por rangos de espesores $H$ y de resistividad $\rho$ validados, a partir de los cuales se modela la resistividad aparente $\rho _{a}$ considerando la apertura de electrodos original $AB/2$, estableciendo de esta manera los datos de entrenamiento \textit{($\rho _{a}$, $AB/2$, $Z$, $\rho$)}, los cuales corresponden a la resistividad aparente simulada $\rho _{a}$ (variable de respuesta), apertura de electrodos $AB/2$, profundidad teórica de muestreo $Z$ y la resistividad $\rho$ del modelo para cada $Z$ teórica.  

El conjunto de datos cuenta con 8 sitios, integrados de 2 a 4 modelos interpretados, sumando un total de 25 SEV's, a partir de cada uno se generan 500 variaciones, en donde los espesores de las unidades definen el rango entre los modelos de un mismo sitio. 

Los datos de entrenamiento se componen de $1640$ filas correspondientes a cada variante de los sitios y $122$ columnas integradas por ID, ID sitio, 30 registros de $\rho _{a}$, 30 registros de $AB/2$, 30 registros de $Z$ y 30 registros de $\rho$; con un total de $196,800$ datos de entrenamiento.

Se realiza el entrenamiento empleando y comparando los modelos Support Vector Machines (SVM), Bayesian Compressive Sensing (BCS) y Random Forests (RF), los cuales se seleccionaron por su tolerancia a las características no paramétricas de los datos, el alto nivel de ruido que presente en la respuesta geoeléctrica, si como una relación compleja entre las variables.%ajustando los hiperparamtros  mediante greed reserch

A partir de los resultados del entrenamiento se realiza pronósticos mientras se compara con datos reales de adquisición de un SEV ($AB/2$ y $Rha$), de igual manera, se realiza una comparación entre las puntuaciones (scorts) de las métricas de los modelos, se analizan los distintos grupos de entrenamiento, evaluando la mejor respuesta y el mejor ajuste en la predicción final.

la aplicación de Machine Learning (ML) representa un gran oportunidad de mejora en el proceso de planeación y adquisición de datos geofísicos, permitiendo establecer un conjunto de entrenamiento con pocos datos de entrada, con la posibilidad de validación en campo durante la adquisición e identificación de patrones, o bien, mediante reconocimiento geológico o expoliación directa.


  %atiene aplicaciones en proyectos de ingeniería civil, vial, petrolera, geohidrológica, minera, etcétera, en donde se requiere conocer el medio sobre el cual se desarrollaran los proyectos, ya sea para definir el estado inicial del medio que sera intervenido, del cual se busca algún beneficio o aprovechamiento

%los metodos geofisicos corresponden a tecnicas no destructivas de exploracion, resultando en un amanera eficiente y rápida para conocer las propiedades físicas del medio, existen distintas técnicas y metodologías de prospección geofísica,  para la seleccion del metodo apropiado se consideran las características del medio y las propiedades físicas del objeto de estudio.

%Una de estas metodologías corresponde a la prospección geoeléctrica, la cual aprovecha las propiedades de conductividad del subsuelo para medir la resistencia la paso del flujo eléctrico a través de las distintas capas del subsuelo, permitiendo conocer la contribución de la resistividad para una sección del medio.

%En este trabajo se aborda la modalida de Sondeo Electrico Vertical, el cual consta de un arreglo de lectrodos incados a tierra, dos de inyeccion de corriente (A y B) y dos de lectura de potencial (M y N) (ver figura 88888), distribuidos geometricamente de acaurdo al tipo de arreglo geoelctrico, durante la adquisicion se incrementa la distancia entre los electrodos abarcando un area de contribucion electrica y consecuentemente profundizando con cada incremento de apertura de electrodos



%\begin{itemize}
%	\item Teorema de muestreo de Nyquist
%	\item Teorema de Shannon-Hartley
%	\item reconocimietno geologico del sitio 
%	\item Espacio espacio disponible
%	\item Topografia del del area de exploracion
%\end{itemize}



