\chapter{INTRODUCCIÓN}
En proyectos de ingeniería civil, vial, petrolera, geohidrológica, minera, etcétera, se requiere conocer el medio sobre el cual se desarrollaran los proyectos, ya sea para conocer el estado inicial del medio que sera intervenido o del cual se busca algún beneficio o remediación que solo sea interpretable mediante exploración indirecta.

Bajo estas circunstancias se emplean métodos indirectos no destructivos para explorar el subsuelo, resultando en un amanera eficiente y rápida para conocer las propiedades físicas del medio, para ello se ejecutan distintas técnicas y metodologías de prospección geofísica, teniendo en cuenta las características del medio y las propiedades físicas del objeto de estudio.

Una de estas metodologías corresponde a la prospección geoeléctrica, la cual aprovecha las propiedades de conductividad del subsuelo para medir la resistencia la paso del flujo eléctrico a través de las distintas capas del subsuelo, permitiendo conocer la contribución de la resistividad para una sección del medio.

Esta información es generada in situ empleando distintas configuraciones de electrodos de lectura y de inyección e corriente, y es en la configuración de estos electrodos donde a partir de la las ecuaciones de flujo eléctrico se describe la forma en que este se dispersa en el subsuelo, permitiendo conocer la profundidad aparente de exploración y la resistividad aparente del área de contribución geoeléctrica.

Para establecer los intervalos de prospeccion convencionalmente se los siguientes criterios:

\begin{itemize}
	\item Teorema de muestreo de Nyquist
	\item Teorema de Shannon-Hartley
	\item 
\end{itemize}



Es aquí donde se pretende establecer una relación de predictibilidad entre un medio conocido y nuevos levantamientos
 


oportunidades de la cuiencia de datos en el análisis en tiempo real de la respuesta 



