\chapter{INTRODUCCIÓN}
la prospeccion geofisic atiene aplicaciones en proyectos de ingeniería civil, vial, petrolera, geohidrológica, minera, etcétera, en donde se requiere conocer el medio sobre el cual se desarrollaran los proyectos, ya sea para definir el estado inicial del medio que sera intervenido, del cual se busca algún beneficio o aprobechamiento, correspondeindo al conjunto de metodos  de exploración indirecta.

los metodos geofisicos corresponden a tecnicas no destructivas de exploracion, resultando en un amanera eficiente y rápida para conocer las propiedades físicas del medio, existen distintas técnicas y metodologías de prospección geofísica,  para la seleccion del metodo apropiado se consideran las características del medio y las propiedades físicas del objeto de estudio.

Una de estas metodologías corresponde a la prospección geoeléctrica, la cual aprovecha las propiedades de conductividad del subsuelo para medir la resistencia la paso del flujo eléctrico a través de las distintas capas del subsuelo, permitiendo conocer la contribución de la resistividad para una sección del medio.

En este trabajo se aborda la modalida de Sondeo Electrico Vertical, el cual consta de un arreglo de lectrodos incados a tierra, los cuales sirven como electrodos de inyeccion (A y B) y lectrodos de lectura (M y N) (ver figura 88888), distribuidos geometricamente en la configuracion definida por el tipo de arreglo geoelctrico, pudiendo ser Wenner, Dipolo Dipolo, Schlumberger, etc..., durante la adquisicion es posible conocer la apaertura de electros (o intervalo de muestreo) y la resitividad aparente del medio 

Para definir los intervalos de muestreo se consideran algunos criterios, como son: 

\begin{itemize}
	\item Teorema de muestreo de Nyquist
	\item Teorema de Shannon-Hartley
	\item reconocimietno geologico del sitio 
	\item Espacio espacio disponible
	\item Topografia del lugar
\end{itemize}

sin embar



Es aquí donde se pretende establecer una relación de predictibilidad entre un medio conocido y nuevos levantamientos
 


oportunidades de la cuiencia de datos en el análisis en tiempo real de la respuesta 



