\chapter{METODOLOGÍA}

%\section{ideas y apuntes}


%\textbf{Adquisición de datos y preprocesamiento: En primer lugar, se realiza la adquisición de datos in situ utilizando el método de Sondeo Eléctrico Vertical (SEV), el cual consiste en medir la resistividad del terreno a diferentes profundidades con un intervalo de muestreo predefinido. Este intervalo se determina de acuerdo con el objetivo de exploración, como la identificación de unidades geológicas, acuíferos, fallas, fracturas o estructuras antropogénicas. Para garantizar que los datos sean adecuados para el análisis, se lleva a cabo un preprocesamiento de los datos, que incluye la limpieza de valores atípicos, normalización de las lecturas y manejo de valores faltantes.}

	El objetivo central de la metodología consiste en evaluar la eficacia de la generación de pronósticos de resistividad aparente a partir del entrenamiento supervisado de los modelos Random Forest, Gradient Boosting Regression y Vector Sopporter Machine.
	
	En ciencia de datos la cadena de procesos de análisis, se estructura y diseña rigurosamente para garantizar la calidad de los datos de entrada, la preparación y diseño de los datos de entrenamiento, la optimización de los modelos y la evaluación de los resultados, así como la comparación de cada modelo empleado.
	
	El proceso metodológico se dividió en 5 etapas:
	
	\begin{enumerate}
		\item Análisis de datos iniciales y simulación
		\item Limpieza y preparación de datos
		\item Modelado y entrenamiento
		\item Validación, optimización y comparación de modelos
		\item Test y evaluación
	\end{enumerate}
	

	\section{Análisis de datos iniciales y simulación}
	
	Durante las etapas de adquisición "in situ", se cuenta con información limitada generada en gabinete y los datos obtenidos durante la adquisición geofísica, estos últimos correspondiendo a profundidades aparentes de exploración, aperturas de electrodos de corriente y potencial, así como a los resultados de resistividad aparente resultantes de cada lectura.
	
	Ya que se cuenta con un numero limitado de variables, $\rho _{A}$, $AB/2$, $Z$ y el modelo de unidades geológicas esperado, así como su valor de resistividad inferido, estos datos conforman los elementos básicos necesarios durante la exploración y por consiguiente se integran como base para el entrenamiento de los modelos.
	
	Antes de poder ser empleadas las variables identificadas, es necesario establecer la fuentes de los datos, definir el formato adecuado y cubrir la dimensión y estructura necesaria en los datos para lograr un entrenamiento efectivo. 
	
		\subsection{Datos iniciales}
		
			Se cuenta con un total de 8 sitios de exploración, los cuales están integrados por al menos con 2 Sondeos Eléctricos Verticales (SEV), en diferentes ambientes geológicos, reuniendo un total de 26 SEV's, correspondientes a sitios de exploración minera, geohidrológica y geotécnica, los cuales se modelaron, interpretaron y validaron mediante exploración por perforación o excavación.
			
			Con el objetivo de reducir el sesgo, no se emplean los datos de resistividad aparente medido in situ, siendo empleados en el test, en su lugar se emplean los modelos validados de cada sitio y sondeo, integrados por datos de espesores de unidades geológicas y las resistividades resultantes del modelo de inversión para cada espesor.
			
			En este punto es importante considerar cuatro aspectos importantes, la heterogeneidad del subsuelo, las condiciones geotécnicas particulares del medio, el principio de equivalencia y la densidad de datos óptimos requeridos por los modelos para su entrenamiento; los dos primeros corresponden a factores cruciales en el comportamiento eléctrico del medio y su respuesta; los últimos dos atañe a la complejidad de interpretación, relacionada con la cantidad de modelos requeridos para el entrenamiento.
			
			Como estrategia, ante estos enormes obstáculos, se opto por una postura centrada en la aleatoriedad de las propiedades resultantes (espesor y resistividad), empleando los rangos respectivos de cada sitio, teniendo en cuanta que los resultados obtenidos corresponden a condiciones geológicas similares, donde se observan mismas unidades, pero con distinta respuesta geoeléctricas.
			
			Al considerar un estado "aleatorio" entre los rangos de valores de las variables evitamos ajustarnos a una distribución pre establecida, evitando así overfitting inicial, lo cual permite entrenar los modelos con todas las variaciones posibles.
			
			 
			
			
			
			
			
			
			
			Para el entrenamiento se emple 
			
			
			
			\subsubsection{Limpieza de datos}
			
		
		
	
		


	\section{Limpieza y preparación de datos}

	\section{Modelado y entrenamiento}

	\section{Validación, optimización y comparación de modelos}
	
	\section{Test y evaluación}
	