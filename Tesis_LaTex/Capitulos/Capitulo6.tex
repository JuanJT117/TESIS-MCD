\chapter{METODOLOGÍA}

%\section{ideas y apuntes}


%\textbf{Adquisición de datos y preprocesamiento: En primer lugar, se realiza la adquisición de datos in situ utilizando el método de Sondeo Eléctrico Vertical (SEV), el cual consiste en medir la resistividad del terreno a diferentes profundidades con un intervalo de muestreo predefinido. Este intervalo se determina de acuerdo con el objetivo de exploración, como la identificación de unidades geológicas, acuíferos, fallas, fracturas o estructuras antropogénicas. Para garantizar que los datos sean adecuados para el análisis, se lleva a cabo un preprocesamiento de los datos, que incluye la limpieza de valores atípicos, normalización de las lecturas y manejo de valores faltantes.}

	El objetivo central de la metodología consiste en evaluar la eficacia de la generación de pronósticos de resistividad aparente a partir del entrenamiento supervisado de los modelos Random Forest, Gradient Boosting Regression y Vector Sopporter Machine.
	
	En ciencia de datos la cadena de procesos de análisis, se estructura y diseña rigurosamente para garantizar la calidad de los datos de entrada, la preparación y diseño de los datos de entrenamiento, la optimización de los modelos y la evaluación de los resultados, así como la comparación de cada modelo empleado.
	
	El proceso metodológico se dividió en 5 etapas:
	
	\begin{enumerate}
		\item Análisis de datos iniciales y simulación
		\item Limpieza y preparación de datos
		\item Modelado y entrenamiento
		\item Validación, optimización y comparación de modelos
		\item Test y evaluación
	\end{enumerate}
	

	\section{Análisis de datos iniciales y simulación}
	
	Durante las etapas de adquisición "in situ", se cuenta con datos limitados, contando unicamente con la información generada en gabinete y los datos obtenidos durante la adquisición geofísica, correspondiendo a profundidades aparentes de exploración, aperturas de electrodos de corriente y potencial, así como los resultados de resistividad aparente resultantes de cada lectura.
	
	Ya que se cuenta con un numero limitado de variables, $\rho _{A}$, $AB/2$, $Z$ y el modelo de unidades geológicas esperada, así como su valor de resistividad asociado, como elementos base para el entrenamiento de los modelos.
	
	Las variables identificadas no pueden integrarse directamente a los datos de entrenamiento, para ello se requiere contar con condiciones de simetría en los datos, así como cubrir o remplazar los datos faltantes mediante técnicas de análisis de datos.
	
	Para la evaluación e los modelos que se implementaran, así como el objetivo de uso, considero el uso de datos procesados y validados en campo, en condiciones y situaciones en las que los datos obtenidos y los modelos presentados pudieron ser validados por exploración directa.
	
	Se cuenta con un total de 8 sitios de exploración, los cuales cuentan al menos con 2 sondeos realizados, en diferentes ambientes geológicos, reuniendo un total de 26 Sondeos Eléctricos Verticales, adicionalmente con el objetivo de reducir el sesgo, no se emplean los datos de resistividad aparente medido in situ, en su lugar se generan modelos sintéticos para el entrenamiento, dejando los datos reales para las para los test y evaluación de los modelos.
	
	%asi como un numero aun mas limitado de muestras para un entrenamiento 
	
	
	
		\subsection{Datos iniciales}
			
			Para los datos de entrada se emplearon levantamientos de SEV, empleando la configuración geométrica Shlumberger, previamente procesados e interpretados, ya sea mediante correlación geológica o con muestreo directo por sondeo de penetración estándar (SPT por sus siglas en inglés), correspondientes a ambiente de deposito sedimentario y flujos volcánico, en ambos caso subyaciendo a unidades sedimentarias recientes.
			
			A partir de estos resultados etiquetados, valores de resistividad aparente, es como de modela variaciones, modificando el espesor de las unidades, ya que cada muestreo de resistividad aparente integra la respuesta conjunta de las unidades que la preceden, es decir las capas geoeléctricas por arriba de la profundidad aparente de exploración, para iguar las condiciones en los modelos que integran los datos de entrenamiento, se emplea la Librería PyGIMLI, la cual esta preparada para realizar esta tarea.
			
			Los datos corresponden a trabajos realizados en en distintas condiciones geológicas, correspondientes a proyectos de exploración hidrológica y minera, se integran un total de 99999 SEV's, procesados, interpretados y validados, a partir de los cuales se generaran las variantes para generar la base de entrenamiento. 
			
			La información particular de nombres de proyectos, localidad, ubicación geográfica o cualquier información que pueda relacionar directamente al propietario del proyecto, son omitidos. 
			
			\subsubsection{Limpieza de datos}
			
			En esta etapa se consideran los siguientes criterios para la selección y limpieza de datos, permitiendo detectar y corregir errores, identificar valores atípicos en el muestreo así como datos inconsistentes.
			
			\begin{description}
				\item[1.- Interpretacion geologica de los perfiles] Deberán incluir interpretación geológica de los resultados de inversión, es decir, es necesario conocer la unidad geológica correspondiente al modelo de resistividad, correspondiendo a etiquetas de datos. 
				\item[2.- Muestreo continuos a intervalos regularres] En caso de que no se cuente con un muestreo adecuado, se integraran los intervalos faltantes, de manera que se modele la señal completa en un muestreo extenso, considerando como mínimo 30 datos de muestreo por sondeo.  
				\item[3.- profundidad de exploracion calculada] Poder identificar durante la adquisición, la profundidad de exploración y los valores de resistividad aparente en el medio, permite detectar variaciones o puntos de inflexión en la curva de la señal.
				\item[4.- Valores de resitividad atipicos] Estos errores por inconsistencia pueden surgir por una mala lectura en campo, al no identificar un cambio de polaridad en el medio.
				\item[5.- Valores duplicados ] identificación de modelos duplicados en los registros.
			\end{description}
		
	
		


	\section{Limpieza y preparación de datos}

	\section{Modelado y entrenamiento}

	\section{Validación, optimización y comparación de modelos}
	
	\section{Test y evaluación}
	