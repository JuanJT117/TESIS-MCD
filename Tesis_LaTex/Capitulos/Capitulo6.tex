\chapter{METODOLOGÍA}

%\section{ideas y apuntes}


%\textbf{Adquisición de datos y preprocesamiento: En primer lugar, se realiza la adquisición de datos in situ utilizando el método de Sondeo Eléctrico Vertical (SEV), el cual consiste en medir la resistividad del terreno a diferentes profundidades con un intervalo de muestreo predefinido. Este intervalo se determina de acuerdo con el objetivo de exploración, como la identificación de unidades geológicas, acuíferos, fallas, fracturas o estructuras antropogénicas. Para garantizar que los datos sean adecuados para el análisis, se lleva a cabo un preprocesamiento de los datos, que incluye la limpieza de valores atípicos, normalización de las lecturas y manejo de valores faltantes.}

	El objetivo central de la metodología consiste en evaluar la eficacia de la generación de pronósticos de resistividad aparente a partir del entrenamiento supervisado de los modelos Random Forest, Gradient Boosting Regression y Vector Sopporter Machine.
	
	En ciencia de datos la cadena de procesos de análisis, se estructura y diseña rigurosamente para garantizar la calidad de los datos de entrada, la preparación y diseño de los datos de entrenamiento, la optimización de los modelos y la evaluación de los resultados, así como la comparación de cada modelo empleado.
	
	El proceso metodológico se dividió en 5 etapas:
	
	\begin{enumerate}
		\item Análisis de datos iniciales y simulación
		\item Limpieza y preparación de datos
		\item Modelado y entrenamiento
		\item Validación, optimización y comparación de modelos
		\item Test y evaluación
	\end{enumerate}
	

	\section{Análisis de datos iniciales y simulación}
	
	Durante las etapas de adquisición "in situ", se cuenta con información limitada generada en gabinete y los datos obtenidos durante la adquisición geofísica, estos últimos correspondiendo a profundidades aparentes de exploración, aperturas de electrodos de corriente y potencial, así como a los resultados de resistividad aparente resultantes de cada lectura.
	
	Ya que se cuenta con un numero limitado de variables, $\rho _{A}$, $AB/2$, $Z$ y el modelo de unidades geológicas esperado, así como su valor de resistividad inferido, estos datos conforman los elementos básicos necesarios durante la exploración y por consiguiente se integran como base para el entrenamiento de los modelos.
	
	Antes de poder ser empleadas las variables identificadas, es necesario establecer la fuentes de los datos, definir el formato adecuado y cubrir la dimensión y estructura necesaria en los datos para lograr un entrenamiento efectivo. 
	
		\begin{table}[h!] % Resultados del modelo de resistividad por sitio. tab:resistividad_sitios
		\centering
		\caption{Resultados del modelo de resistividad por sitio.}
		\label{tab:resistividad_sitios}
		
		% resizebox ajustará la tabla al ancho del texto de la página.
		\resizebox{\textwidth}{!}{%
			
			% \scriptsize establece el tamaño de letra base que solicitaste.
			\scriptsize 
			
			% Usamos 'S' de siunitx para alinear todas las columnas numéricas.
			\begin{tabular}{l S S S S S S S S}
				\toprule
				% --- Encabezados (usamos \shortstack para que se dividan en dos líneas) ---
				{\textbf{Modelo}} & 
				{\shortstack{\textbf{Sitio}}} & 
				{\shortstack{\textbf{Espesor} \\ \textbf{1}}} & 
				{\shortstack{\textbf{Espesor} \\ \textbf{2}}} & 
				{\shortstack{\textbf{Espesor} \\ \textbf{3}}} & 
				{\shortstack{\textbf{Resistividad} \\ \textbf{1}}} & 
				{\shortstack{\textbf{Resistividad} \\ \textbf{2}}} & 
				{\shortstack{\textbf{Resistividad} \\ \textbf{3}}} & 
				{\shortstack{\textbf{Resistividad} \\ \textbf{4}}} \\
				\midrule
				
				% --- Datos de la tabla ---
				S1-LN-EBSA   & 1 & 6.1   & 60.7   & 131.3  & 150.0    & 60.0     & 804.5     & 965.4    \\
				S2-LN-EBSA   & 1 & 6.4   & 78.4   & 113.5  & 196.01   & 60.0     & 4765.06   & 5718.0 \\
				S3-LN-EBSA   & 1 & 6.0   & 70.0   & 122.0  & 437.2    & 90.0     & 100.0     & 120.0      \\
				S1-AMEBSA    & 2 & 2.4   & 8.5    & 45.5   & 56.6     & 350.0    & 11000.0   & 13200.0    \\
				S2-AMEBSA    & 2 & 1.3   & 4.5    & 66.0   & 56,0     & 350.0    & 1500.0    & 1800.0     \\
				S3-AMEBSA    & 2 & 3.0   & 23.0   & 32.0   & 95.0     & 1500.0   & 4800.0    & 5760.0     \\
				S4-AMEBSA    & 2 & 3.0   & 9.0    & 63.0   & 4.0      & 45.0     & 3400.0    & 4080.0     \\
				S5-AMEBSA    & 2 & 3.5   & 9.0    & 41.0   & 48.0     & 460.0    & 2300.0    & 2760.0     \\
				S1-BMEBSA    & 3 & 3.0   & 7.0    & 20.0   & 125      & 3350.0   & 4020.0    &          \\
				S2-BMEBSA    & 3 & 4.0   & 6.0    & 20.0   & 2.2      & 1500.0   & 1800.0    &          \\
				S1-MFA-VER   & 4 & 2.9   & 19.9   & 37.2   & 86.7     & 13.4     & 195.4     & 234.4   \\
				S2-MFA-VER   & 4 & 1.7   & 15.7   & 42.6   & 21.8     & 1.9      & 80.0      & 96.0       \\
				S3-MFA-VER   & 4 & 1.7   & 5.6    & 52.7   & 26.1     & 77.8     & 32.5      & 39.0       \\
				S1-OA-VER    & 5 & 1.6   & 25.0   & 33.4   & 2.1      & 6.9      & 77.0      & 92.4     \\
				S2-OA-VER    & 5 & 0.7   & 6.4    & 43.6   & 38.0     & 1.0      & 70.0      & 84.0       \\
				S3-OA-VER    & 5 & 2.0   & 7.9    & 59.0   & 1.0      & 15.7     & 70.0      & 84.0       \\
				S1-LP-L-NL   & 6 & 2.36  & 15.8   & 53.8   & 111.6    & 1676.5   & 1271.4    & 1525.6  \\
				S2-LP-L-NL   & 6 & 4.0   & 20.7   & 47.3   & 398.77   & 690.16   & 1013.7    & 1216.4  \\
				S1-EGH-OLV   & 7 & 3.0   & 14.0   & 80.0   & 56.0     & 264.0    & 85.0      & 129.0      \\
				S2-EGH-OLV   & 7 & 10.0  & 26.0   & 204.0  & 17.0     & 48.0     & 360.0     & 432.0      \\
				S3-EGH-OLV   & 7 & 40.0  & 20.0   & 138.0  & 750.0    & 100.0    & 85.0      & 102.0      \\
				S1-ZR-P-EBSA & 8 & 3.5   & 12.0   & 44.5   & 3.0      & 20.0     & 2000.0    & 2400.0     \\
				S2-ZR-P-EBSA & 8 & 4.0   & 6.0    & 56.0   & 350.0    & 1300.0   & 2600.0    & 3120.0     \\
				S3-ZR-P-EBSA & 8 & 8.0   & 37.0   & 20.0   & 250.0    & 2600.0   & 1800.0    & 2160.0     \\
				S4-ZR-P-EBSA & 8 & 5.6   & 13.0   & 20.0   & 1300.0   & 3200.0   & 1800.0    & 5000.0     \\
				\bottomrule
			\end{tabular}%
		} % Cierre del resizebox<
	\end{table}
	
	\clearpage
	
		\subsection{Datos iniciales}
		
			Se cuenta con un total de 8 sitios de exploración, los cuales están integrados por al menos con 2 Sondeos Eléctricos Verticales (SEV), en diferentes ambientes geológicos, reuniendo un total de 26 SEV's, correspondientes a sitios de exploración minera, geohidrológica y geotécnica, los cuales se modelaron, interpretaron y validaron mediante exploración por perforación o excavación (ver tabla \ref{tab:resistividad_sitios}).
			
			Al analizar los histogramas correspondientes a los espesores, se identifica en los tres gráficos (ver figura \ref{fig:hvi1}) una marcada ocurrencia de unidades con espesores pequeños en cada conjunto; en el grafico Espesor\_1 se asociado a la mayor ocurrencia de unidades poco desarrolladas con estratos geológicos sedimentarios superficiales, mientras que a mayor profundidad, Espesor\_2 y Espesor\_3, la ocurrencia de unidades geológicas con mayor desarrollo es mas recurrente, sin embargo esto no puede considerarse una norma al considerar la complejidad geológica de cada sitio particular.\\
			
			
			\begin{figure}[h!] % histograma_variables_Iniciales.png
				\centering
				\includegraphics[width=15cm]{Imagenes/6.png}
				\caption[Histogramas de los espesores identificados en los modelos de interpretación geofísica]{Histogramas de los espesores en los modelos interpretados.}
				\label{fig:hvi1}
			\end{figure}
			
			\clearpage
			
			Mientras que los histogramas de resistividad presentan una ocurrencia mayor de valores bajos con colas largas y pesadas, es decir, se observan pocos espesores con un alto valor de resistividad y muchos con bajo valor (ver figura \ref{fig:hvi2}), este comportamiento esta directamente ligado directamente a la las unidades interpretadas que fueron muestreadas, y corresponde a una propiedad misma del medio explorado.
			
			\begin{figure}[h!] % histograma_variables_Iniciales.png
				\centering
				\includegraphics[width=15cm]{Imagenes/7.png}
				\caption[Histogramas de las resistividades identificadas en los modelos de inversión geofísica]{Histogramas de las resistividades en los modelos interpretados.}
				\label{fig:hvi2}
			\end{figure}
			
			En la figura \ref{fig:bp-s1} se observa la distribución general entre los distintos sitio; en el primer grafico correspondiente al grupo 'Espesor\_1' se identifica en el sitio 7 una alta dispersión así como una marcada asimetría en los datos dada la posición del Rango Intercuartílico mientras que en el resto se observa una marcada concentración en una posición baja; en el segundo grupo, 'Espesor\_2', se identifica un ligero incremento en la dispersión general de los datos, manteniendo la simetría observada anteriormente; en el ultimo grupo de 'Espesor\_3' una variación mayor en la posición de los cuartiles, manteniendo simetría con una mayor dispersión en el rango de amplitud de espesor.
			
			los comportamientos observados en la figura \ref{fig:bp-s1} son los esperados para conjuntos de modelos interpretados en condiciones geológicas similares, mas no idénticas, mostrando un relaciona intrínseca de la heterogeneidad de los espesores identificados con la ambigüedad en la interpretación de los modelos de inversión.
			
			\begin{figure}[h!] % Box-plot 8.png  fig:bp-s1
				\centering
				\includegraphics[width=15cm]{Imagenes/8.png}
				\caption[Boxplot de la distribución global de espesores por sitio]{Boxplot de la distribución global de espesores por sitio.}
				\label{fig:bp-s1}
			\end{figure}
			
			En la figura \ref{fig:bp-s2} se identifica una marcada similitud de dispersión, posición y simetría entre los gráficos de distribución de resistividad 3 y 4, dicha similitud cualitativa se asocia a la aproximación de valores resistivos en los limites del modelo de inversión, individualmente en los gráficos de distribución de resistividad 1, 2 y 3 se observa un aumento progresivo de la dispersión entre los gráficos, se observa que los sondeos 4, 5 y 6 presentan una posición estable en los tres gráficos, esta falta de dispersión se asocia a la baja profundidad de los sondeos y la presencia de un ambiente geológico muy similar en todo el rango de exploración.
			
			\begin{figure}[h!] % Box-plot 9.png  fig:bp-s2
				\centering
				\includegraphics[width=15cm]{Imagenes/9.png}
				\caption[Boxplot de la distribución global de resistividad por sitio]{Boxplot de la distribución global de resistividad por sitio.}
				\label{fig:bp-s2}
			\end{figure}
			
			Al realizar el análisis de componentes principales (PCA) identificamos que los primeros 4 componentes (Espesor 1, Espesor 2, Espesor 3 y Resistividad 1) describen casi el 92\% de la variabilidad de los datos, muy cerca del umbral del 95\%, siendo prácticamente no representativa la componente 7 (Resistividad 4).
			
			\begin{enumerate}
				\item PC1: 39.41\% (Espesor 1)
				\item PC2: 28.88\% (Espesor 2)
				\item PC3: 13.72\% (Espesor 3)
				\item PC4: 9.74\% (Resistividad 1)
				\item PC5: 4.58\% (Resistividad 2)
				\item PC6: 3.61\% (Resistividad 3)
				\item PC7: 0.06\% (Resistividad 4)
			\end{enumerate}
			
			\begin{figure}[h!] % Box-plot 9.png  fig:bp-s2
				\centering
				\includegraphics[width=15cm]{Imagenes/PCA-Analisis_Componentes_Principales.png}
				\caption[Análisis de componentes principales en datos base]{Análisis de componentes principales obteniendo la siguiente varianza por componente: 0.3941 0.2888 0.1372 0.0974 0.0458 0.0361 0.0006.}
				\label{fig:PCA-1}
			\end{figure}
			
			Al analizar los resultados cualitativos y cuantitativos resultantes de las evaluaciones estadísticas, se observa que las variables correspondientes a los 8 sitios corresponden a datos no relacionados y lo suficientemente independientes, este comportamiento se confirma al observar las matriz de correlación entre variables (ver figura \ref{fig:MCVI}), observando una valor de correlación general menor a 0.52, a excepción del valor obtenido entre la variable  Resistividad\_3 y Resistividad\_4.   
			
			\begin{figure}[h!] % Box-plot 9.png  fig:bp-s2
				\centering
				\includegraphics[width=15cm]{Imagenes/MC-DI.png}
				\caption[Matriz de correlación de variables iniciales]{Matriz de correlación de variables iniciales, correspondientes a los conjuntos de datos modelos interpretados.}
				\label{fig:MCVI}
			\end{figure}
			 
			
	\section{Preparación de datos}
			
			el volumen de datos iniciales, aunque significativo localmente para la interpretación de cada sitio, es insuficiente para establece una base de entrenamiento robusta, por lo que es imprescindible contar con las cantidad de datos suficiente para el entrenamiento.
			
			Para lograr el volumen de datos necesarios se implementa la generación de SEV's sintéticos, en esta tarea se emplea la librería PyGIMLI, publicada por \cite{Ruecker2017}, la cual permite realizar una simulación de sondeo a partir de unidades de espesor con un valor de resistividad establecido y la relación geométrica de apertura de los electrodos de corriente.
			
			Al contar con datos de espesores y resistividad producto de una interpretación geoeléctrica, cuyos valores (ver tabla \ref{tab:resistividad_sitios}) fueron validadas directamente para cada sondeo, se cuenta con la base necesaria para generar modelos sintéticos de respuesta geoeléctrica; empleando los valores de apertura de electrodos ($AB/2$) de los sondeos originales.
			
			En este punto es importante considerar cuatro aspectos, la heterogeneidad del subsuelo, las condiciones geotécnicas particulares del medio, el principio de equivalencia y la densidad de datos requeridos por los modelos para su entrenamiento; los dos primeros corresponden a factores cruciales que determinan el comportamiento eléctrico del medio y su respuesta; los últimos dos atañe a la complejidad de interpretación, relacionada con la cantidad de modelos requeridos para el entrenamiento y el infinito numero de modelos de inversión de ajuste de la señal resultante.
			
			Como estrategia se opto por una postura centrada en la aleatoriedad de las propiedades evaluadas y validadas, empleando los rangos máximos y mínimos de espesor y resistividad de cada sitio, teniendo en cuanta que los resultados obtenidos corresponden a condiciones geológicas similares pero con distinta respuesta geoeléctricas, cubriendo el aspecto de heterogeneidad del subsuelo y el principio de equivalencia, con rangos aleatorios de resistividad y espesor respectivamente (ver tabla \ref{tab:espesores_sitio}). 
			
			\begin{table}[h!]
				\centering
				\tiny
				\setlength{\tabcolsep}{6pt} 
				
				\caption{Rango de espesores (mínimo y máximo) registrados por sitio.}
				\label{tab:espesores_sitio}
				\begin{tabular}{ 
						c                   % Sitio
						l                   % Espesor (Nombre)
						S[table-format=3.2] % Mínimo (alineado decimal)
						S[table-format=3.2] % Máximo (alineado decimal)
					}
					\toprule
					\textbf{Sitio} & \textbf{variable} & {\textbf{Mínimo}} & {\textbf{Máximo}} \\
					\midrule
					
					1 & Espesor 1 & 6.00 & 6.40 \\
					& Espesor 2 & 60.70 & 78.47 \\
					& Espesor 3 & 113.53 & 131.30 \\
					\addlinespace
					2 & Espesor 1 & 1.30 & 3.50 \\
					& Espesor 2 & 4.50 & 23.00 \\
					& Espesor 3 & 32.00 & 66.00 \\
					\addlinespace
					3 & Espesor 1 & 3.00 & 4.00 \\
					& Espesor 2 & 6.00 & 7.00 \\
					& Espesor 3 & 20.00 & 20.00 \\
					\addlinespace
					4 & Espesor 1 & 1.70 & 2.90 \\
					& Espesor 2 & 5.60 & 19.90 \\
					& Espesor 3 & 37.20 & 52.70 \\
					\addlinespace
					5 & Espesor 1 & 0.70 & 2.00 \\
					& Espesor 2 & 6.40 & 25.00 \\
					& Espesor 3 & 33.40 & 59.00 \\
					\addlinespace
					6 & Espesor 1 & 2.36 & 4.00 \\
					& Espesor 2 & 15.80 & 20.70 \\
					& Espesor 3 & 47.30 & 53.80 \\
					\addlinespace
					7 & Espesor 1 & 3.00 & 40.00 \\
					& Espesor 2 & 14.00 & 26.00 \\
					& Espesor 3 & 80.00 & 204.00 \\
					\addlinespace
					8 & Espesor 1 & 3.50 & 8.00 \\
					& Espesor 2 & 6.00 & 37.00 \\
					& Espesor 3 & 20.00 & 56.00 \\
					
					\bottomrule
				\end{tabular}
			\end{table}
			
			
			Al considerar un estado \textit{aleatorio} entre los rangos de valores de las variables evitamos el ajuste a una distribución no representativa, evadiendo así el overfitting inicial, lo cual permite entrenar los modelos con todas las variaciones aleatorias posibles.
			
			Finalmente las variaciones sinbteticas 
			
			
		
			
	\section{Modelado y entrenamiento}
	
			características ajustes de cada modelo empleado y evaluación $r^{2}$

	\section{Validación, optimización y comparación de modelos}
	
	
	
	\section{Test y evaluación}
	