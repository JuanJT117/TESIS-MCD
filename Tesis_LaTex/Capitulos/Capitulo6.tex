\chapter{METODOLOGÍA}

%\section{ideas y apuntes}


%\textbf{Adquisición de datos y preprocesamiento: En primer lugar, se realiza la adquisición de datos in situ utilizando el método de Sondeo Eléctrico Vertical (SEV), el cual consiste en medir la resistividad del terreno a diferentes profundidades con un intervalo de muestreo predefinido. Este intervalo se determina de acuerdo con el objetivo de exploración, como la identificación de unidades geológicas, acuíferos, fallas, fracturas o estructuras antropogénicas. Para garantizar que los datos sean adecuados para el análisis, se lleva a cabo un preprocesamiento de los datos, que incluye la limpieza de valores atípicos, normalización de las lecturas y manejo de valores faltantes.}

	El objetivo central de la metodología consiste en evaluar la eficacia de la generación de pronósticos de resistividad aparente a partir del entrenamiento supervisado de los modelos Random Forest, Gradient Boosting Regression y Vector Sopporter Machine.
	
	En ciencia de datos la cadena de procesos de análisis, se estructura y diseña rigurosamente para garantizar la calidad de los datos de entrada, la preparación y diseño de los datos de entrenamiento, la optimización de los modelos y la evaluación de los resultados, así como la comparación de cada modelo empleado.
	
	El proceso metodológico se dividió en 5 etapas:
	
	\begin{enumerate}
		\item Análisis de datos iniciales y simulación
		\item Limpieza y preparación de datos
		\item Modelado y entrenamiento
		\item Validación, optimización y comparación de modelos
		\item Test y evaluación
	\end{enumerate}
	

	\section{Análisis de datos iniciales y simulación}
	
	Durante las etapas de adquisición "in situ", se cuenta con información limitada generada en gabinete y los datos obtenidos durante la adquisición geofísica, estos últimos correspondiendo a profundidades aparentes de exploración, aperturas de electrodos de corriente y potencial, así como a los resultados de resistividad aparente resultantes de cada lectura.
	
	Ya que se cuenta con un numero limitado de variables, $\rho _{A}$, $AB/2$, $Z$ y el modelo de unidades geológicas esperado, así como su valor de resistividad inferido, estos datos conforman los elementos básicos necesarios durante la exploración y por consiguiente se integran como base para el entrenamiento de los modelos.
	
	Antes de poder ser empleadas las variables identificadas, es necesario establecer la fuentes de los datos, definir el formato adecuado y cubrir la dimensión y estructura necesaria en los datos para lograr un entrenamiento efectivo. 
	
		\subsection{Datos iniciales}
		
			Se cuenta con un total de 8 sitios de exploración, los cuales están integrados por al menos con 2 Sondeos Eléctricos Verticales (SEV), en diferentes ambientes geológicos, reuniendo un total de 26 SEV's, correspondientes a sitios de exploración minera, geohidrológica y geotécnica, los cuales se modelaron, interpretaron y validaron mediante exploración por perforación o excavación.
			
			Al contar con una interpretación geoeléctrica, cuyos espesores y resistividades (ver tabla \ref{tab:resistividad_sitios}) fueron validadas directamente para cada sondeo, se emplean estas propiedades, bien acotadas, para generar modelos sintéticos de respuesta geoeléctrica, a partir de la librería pyGIMLI publicada por \cite{Ruecker2017}, reservando los datos originales de resistividad de cada sitio para la etapa de evaluación. 
			
			\begin{table}[h!] % Resultados del modelo de resistividad por sitio. tab:resistividad_sitios
				\centering
				\caption{Resultados del modelo de resistividad por sitio.}
				\label{tab:resistividad_sitios}
				
				% resizebox ajustará la tabla al ancho del texto de la página.
				\resizebox{\textwidth}{!}{%
					
					% \scriptsize establece el tamaño de letra base que solicitaste.
					\scriptsize 
					
					% Usamos 'S' de siunitx para alinear todas las columnas numéricas.
					\begin{tabular}{l S S S S S S S S}
						\toprule
						% --- Encabezados (usamos \shortstack para que se dividan en dos líneas) ---
						{\textbf{Modelo}} & 
						{\shortstack{\textbf{Sitio}}} & 
						{\shortstack{\textbf{Espesor} \\ \textbf{1}}} & 
						{\shortstack{\textbf{Espesor} \\ \textbf{2}}} & 
						{\shortstack{\textbf{Espesor} \\ \textbf{3}}} & 
						{\shortstack{\textbf{Resistividad} \\ \textbf{1}}} & 
						{\shortstack{\textbf{Resistividad} \\ \textbf{2}}} & 
						{\shortstack{\textbf{Resistividad} \\ \textbf{3}}} & 
						{\shortstack{\textbf{Resistividad} \\ \textbf{4}}} \\
						\midrule
						
						% --- Datos de la tabla ---
						S1-LN-EBSA   & 1 & 6.1   & 60.7   & 131.3  & 150    & 60     & 804.5   & 965.4    \\
						S2-LN-EBSA   & 1 & 6.4   & 78.47  & 113.53 & 196.01 & 60     & 4765.06 & 5718.072 \\
						S3-LN-EBSA   & 1 & 6     & 70     & 122    & 437.2  & 90     & 100     & 120      \\
						S1-AMEBSA    & 2 & 2.4   & 8.5    & 45.5   & 56.6   & 350    & 11000   & 13200    \\
						S2-AMEBSA    & 2 & 1.3   & 4.5    & 66     & 56     & 350    & 1500    & 1800     \\
						S3-AMEBSA    & 2 & 3     & 23     & 32     & 95     & 1500   & 4800    & 5760     \\
						S4-AMEBSA    & 2 & 3     & 9      & 63     & 4      & 45     & 3400    & 4080     \\
						S5-AMEBSA    & 2 & 3.5   & 9      & 41     & 48     & 460    & 2300    & 2760     \\
						S1-BMEBSA    & 3 & 10    &        & 20     & 125    & 3350   & 4020    &          \\
						S2-BMEBSA    & 3 & 10    &        & 20     & 2.2    & 1500   & 1800    &          \\
						S1-MFA-VER   & 4 & 2.9   & 19.9   & 37.2   & 86.7   & 13.4   & 195.4   & 234.48   \\
						S2-MFA-VER   & 4 & 1.7   & 15.7   & 42.6   & 21.8   & 1.9    & 80      & 96       \\
						S3-MFA-VER   & 4 & 1.7   & 5.6    & 52.7   & 26.1   & 77.8   & 32.5    & 39       \\
						S1-OA-VER    & 5 & 1.6   & 25     & 33.4   & 2.1    & 6.9    & 77      & 92.4     \\
						S2-OA-VER    & 5 & 0.7   & 6.4    & 43.6   & 38     & 1      & 70      & 84       \\
						S3-OA-VER    & 5 & 2     & 7.9    & 59     & 1      & 15.7   & 70      & 84       \\
						S1-LP-L-NL   & 6 & 2.36  & 15.8   & 53.8   & 111.6  & 1676.5 & 1271.4  & 1525.68  \\
						S2-LP-L-NL   & 6 & 4     & 20.7   & 47.3   & 398.77 & 690.16 & 1013.7  & 1216.44  \\
						S1-EGH-OLV   & 7 & 3     & 14     & 80     & 56     & 264    & 85      & 129      \\
						S2-EGH-OLV   & 7 & 10    & 26     & 204    & 17     & 48     & 360     & 432      \\
						S3-EGH-OLV   & 7 & 40    & 20     & 138    & 750    & 100    & 85      & 102      \\
						S1-ZR-P-EBSA & 8 & 3.5   & 12     & 44.5   & 3      & 20     & 2000    & 2400     \\
						S2-ZR-P-EBSA & 8 & 4     & 6      & 56     & 350    & 1300   & 2600    & 3120     \\
						S3-ZR-P-EBSA & 8 & 8     & 37     & 20     & 250    & 2600   & 1800    & 2160     \\
						S4-ZR-P-EBSA & 8 & 5.6   & 13     & 20     & 1300   & 3200   & 1800    & 5000     \\
						\bottomrule
					\end{tabular}%
				} % Cierre del resizebox<
			\end{table}
			
			Al analizar los histogramas de las variables, espesor y resistividad, se identifica una marcada tendencia de bajos espesores respectivos a cada conjunto, lo que indica una falta de variabilidad en los datos (ver figura \ref{fig:hvi}), este mismo patrón se observa en los histogramas de resistividad, correspondiendo al comportamiento general de los datos. 
			
			\begin{figure}[h!] % histograma_variables_Iniciales.png
				\centering
				\includegraphics[width=15cm]{Imagenes/histograma_variables_Iniciales.png}
				\caption[Histograma de las variables de los modelos de interpretación geofísica]{Histograma correspondientes a las variables de los modelos de interpretación geofísica de cada sitio.}
				\label{fig:hvi}
			\end{figure}
			
			Podemos observar en los grafico Box-plot de las variables por sitio (ver figura), una concentración de unidades de poco espesor, este patrón, guardando relación al considerar el ambiente geológico de cada modelo geoeléctrico, es decir encontramos espesores cortos en ambientes sedimentarios y una mayor distribución de los mismos en ambientes mixtos, lo cual corresponde a lo esperado para cada situación; se identifica la dispersión de los valores de resistividad presentan mayor variabilidad en los bloques Resistividad 1 y  2, mientras que las ultimas dos guardan una dispersión similar,
			
			\begin{figure}[h!] % histograma_variables_Iniciales.png
				\centering
				\includegraphics[width=13cm]{Imagenes/box-plot-sitio.png}
				\caption[Histograma de las variables de los modelos de interpretación geofísica]{Histograma correspondientes a las variables de los modelos de interpretación geofísica de cada sitio.}
				\label{fig:bp-s}
			\end{figure}
			
			A partir del análisis estadístico básico de cada sitio a si como de las variables de espesor y resistividad, se identifican patrones globales de distribución, en clasificaciones por sitio como por variable, así como patrones característicos de cada sitio, de igual manera se analiza la correlación entre sitios a fin de identificar posibles 
			
	\section{Limpieza y preparación de datos}
			
			En este punto es importante considerar cuatro aspectos importantes, la heterogeneidad del subsuelo, las condiciones geotécnicas particulares del medio, el principio de equivalencia y la densidad de datos requeridos por los modelos para su entrenamiento; los dos primeros corresponden a factores cruciales que determinan en el comportamiento eléctrico del medio y su respuesta; los últimos dos atañe a la complejidad de interpretación, relacionada con la cantidad de modelos requeridos para el entrenamiento.
			
			Como estrategia, ante estos enormes obstáculos, se opto por una postura centrada en la aleatoriedad de las propiedades resultantes (espesor y resistividad), empleando los rangos respectivos de cada sitio, teniendo en cuanta que los resultados obtenidos corresponden a condiciones geológicas similares, donde se observan mismas unidades, pero con distinta respuesta geoeléctricas.
			
			Al considerar un estado "aleatorio" entre los rangos de valores de las variables evitamos el ajuste a una distribución no representativa, evitando así overfitting inicial, lo cual permite entrenar los modelos con todas las variaciones posibles.
			
			
		
			
	\section{Modelado y entrenamiento}
	
			caracteristicas ajustes de cada modelo empleadoy evaluacion $r^{2}$

	\section{Validación, optimización y comparación de modelos}
	
	\section{Test y evaluación}
	