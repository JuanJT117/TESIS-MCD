\chapter{JUSTIFICACIÓN}
Existen múltiples aplicaciones de ML y DL en el procesamiento, modelado e interpretación geofísica (INSERTAR REFERENCIAS), así como aplicaciones directas por software en las disciplinas mas comerciales (Sísmica de exploración de hidrocarburos PETREL, Shlumberger,  INSERTAR REFERENCIA).

Es de esperar que al implementar ciencias de datos en diversas áreas de la ciencia, se optimicen aspectos básicos, mejorando la analítica e interpretación de resultados, o bien, modelando predicciones de tendencias.

En este sentido, es oportuna la implementación de modelos de regresión y clasificación durante la adquisición de datos, en particular al ejecutar muestreo de Sondeo Eléctrico Vertical (SEV, VES por sus siglas en inglés), el proceso de adquisición \textit{in situ} consta de un intervalo de muestreo predefinido, el cual está acotado de acuerdo al objetivo de exploración, este análisis previo es un factor determinante, debido a que establecer un intervalo de muestreo correcto permite identificar el objetivo, unidades geológicas, acuíferos, fallas, fracturas, estructuras antropogenicas, etc.. , es decir mantener un intervalo de muestreo menor a la frecuencia de ocurrencia del objetivo de estudio, por lo que el éxito de la exploración dependerá en su totalidad de la planeación previa de la adquisición.

De manera que un modelo que permita clasificar las lecturas y generar una regresión para proponer muestreos adicionales \textit{in situ} tendría la ventaja de optimizar y mejorar la calidad de la adquisición y muestreo de las frecuencias deseadas.
