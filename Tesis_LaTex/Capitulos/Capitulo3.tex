\chapter{JUSTIFICACIÓN}
Existen múltiples aplicaciones de ML y DL en el procesamiento, modelado e interpretación geofísica (INSERTAR REFERENCIAS), así como aplicaciones directas por software en las disciplinas mas comerciales (Sísmica de exploración de hidrocarburos PETREL, Shlumberger,  INSERTAR REFERENCIA).

Es de esperar que al implementara ciencias de datos mejoraren aspectos básicos, así como especializados, en diversas áreas de las ciencias, mejorando la analítica e interpretación de resultados, o bien, modelando predicciones.

En este sentido es oportuna la implementación de modelos de regresión y clasificación durante la adquisición de datos de un Sondeo Eléctrico Vertical (SEV, VES por sus siglas en ingles), el proceso de adquisición \textit{in situ} consta de un intervalo de muestreo predefinido, el cual esta acotado de acuerdo al objetivo de exploración, siendo el principal factor establecer un intervalo de muestreo que permita identificar las unidades geológicas de interés, es decir mantener un intervalo de muestreo menor a la frecuencia de ocurrencia de las unidades, por lo que el éxito de la exploración dependerá en su totalidad de la planeación previa de la adquisición.

De manera que un modelo que permita clasificar las lecturas y generar una regresión para proponer muestreos adicionales \textit{in situ} tendría la ventaja de optimizar y mejorar la calidad de la adquisición y muestreo de las frecuencias deseadas.
