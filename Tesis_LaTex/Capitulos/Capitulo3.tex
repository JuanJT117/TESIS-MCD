\chapter{JUSTIFICACIÓN}
Existen múltiples aplicaciones de ML y DL en el procesamiento, modelado e interpretación geofísica \citep{li2024, liu2020,el2001,wrona2018}, algunas de estas aplicaciones corresponden a implementaciones académicas, al igual es posible identificar aplicaciones comerciales implementadas en software principalmente de exploración sísmica de hidrocarburos \citep{diaferia2024high, panebianco2024automated}.


%Es de esperar que al implementar ciencias de datos en diversas áreas de la ciencia, se optimicen aspectos básicos, mejorando la analítica e interpretación de resultados, o bien, modelando predicciones de tendencias.

La aplicación de herramientas de ML durante la ejecutar de muestreo de Sondeo Eléctrico Vertical (SEV), en particular durante el proceso de adquisición \textit{in situ}, presenta la posibilidad de evaluar y ampliar el intervalo de muestreo original, mejorando la adquisición tradicional, permitiendo evaluar el muestreo contra la regresiones y establecer nuevos intervalos de oportunidad.

Durante el muestreo tradicional los intervalos se diseñan considerando el objetivo de exploración (idealmente), este análisis previo es un factor determinante, en esta etapa se establece el intervalo de muestreo que permitiría identificar el objeto de exploración, este objeto de exploración pueden ser unidades geológicas, acuíferos, fallas, zonas de fracturas, estructuras antropogénicas, etc.

De manera general se busca mantener un intervalo de muestreo menor a la frecuencia de ocurrencia del objetivo de estudio, por lo que el éxito de la exploración dependerá en su totalidad de la planeación previa de la adquisición, lo que implicaría conocer previamente la conformación, distribución y espesor de cada unidad, lo que resulta irrisorio.

De manera que un modelo entrenado permita generar múltiples regresiones e identificar regiones de interés no cubiertas por el muestreos y de esta manera generar puntos adicionales \textit{in situ} optimizando la adquisición de datos e impactando en la calidad del muestreo , mejorando el acotamiento de las frecuencias identificadas.

Pese a existir aplicaciones de ML implementadas en geofísica, no se identifica alguna enfocada este esta problemática en particular, sin embargo hay aplicaciones en otros campos de estudio con un enfoque en el muestreo y clasificación de datos no paramétricos \citep{entezami2022non, bkassiny2013multidimensional, shi2021non} empleando técnicas como Dirichlet Process Mixture Model (DPMM), radial basis function network (RBFN),  Multiple Point Statistics (MPS) y Bayesian Compressive Sensing (BCS).

Para poder abordar la problemática se requiere de un modelo robusto ante el ruido, que permita trabajar con datos no paramétricos, establecer un modelo mediante entrenamiento supervisado, dada la complejidad de interpretación, por consiguiente su etiquetado, capas de realizar regresiones a partir de entrenamientos previos y permita realizar predicciones de valores de resistividad. Estas condiciones son cubiertas por dos modelos Bayesian Compressive Sensing (BCS) y Random Forests (RF).

Se Considera ademas de esto la facilidad de implementar e interpretar los resultados, así como el consumo de recursos, por lo que se opta por emplear el modelo Random Forests, así como tolera datos faltantes.
