\chapter{JUSTIFICACIÓN}
Existen múltiples aplicaciones de ML y DL en el procesamiento, modelado e interpretación geofísica \citep{li2024, liu2020,el2001,wrona2018}, algunas de estas aplicaciones corresponden a implementaciones académicas, al igual que comerciales, siendo implementadas en software principalmente de exploración sísmica de hidrocarburos \citep{diaferia2024high, panebianco2024automated}.


%Es de esperar que al implementar ciencias de datos en diversas áreas de la ciencia, se optimicen aspectos básicos, mejorando la analítica e interpretación de resultados, o bien, modelando predicciones de tendencias.

La aplicación de herramientas de ML durante la ejecutar de muestreo de Sondeo Eléctrico Vertical (SEV), en particular durante el proceso de adquisición \textit{in situ}, presenta la posibilidad de evaluar y ampliar el intervalo de muestreo original, mejorando la adquisición tradicional, realizando una comparación del muestreo contra la regresiones a fin de establecer nuevos intervalos de apertura de electrodos no considerados previamente.

Durante el muestreo tradicional los intervalos se diseñan considerando el objetivo de exploración (idealmente), este análisis previo es un factor determinante, en esta etapa permitiendo establecer el intervalo de muestreo que permitirá identificar el objeto de exploración, este objeto o anomalía puede asociarse a unidades geológicas, acuíferos, fallas, zonas de fracturas, estructuras antropogénicas, infraestructura moderna, etc.

De manera general se busca mantener un intervalo de muestreo menor a la frecuencia de ocurrencia del objetivo de estudio, por lo que el éxito de la exploración dependerá en su totalidad de la planeación previa de la adquisición, lo que implicaría conocer previamente la conformación, distribución y espesor de cada unidad, siendo no viable, por lo que la interpretación puede presentar una alta ambigüedad.

De manera que un modelo entrenado permita generar múltiples modelos e identificar regiones de interés no cubiertas por el muestreos a través de predicciones y de esta manera generar puntos adicionales \textit{in situ} optimizando la adquisición de datos e impactando en la calidad del muestreo, mejorando el acotamiento de la respuesta geoeléctrica identificada.

Pese a existir aplicaciones de ML implementadas en geofísica, no se identifica alguna enfocada este esta problemática en particular, sin embargo hay ejemplos en otros campos de estudio con un enfoque en el muestreo y clasificación de datos no paramétricos \citep{entezami2022non, bkassiny2013multidimensional, shi2021non} empleando técnicas como Dirichlet Process Mixture Model (DPMM), radial basis function network (RBFN),  Multiple Point Statistics (MPS) y Bayesian Compressive Sensing (BCS).

Para poder abordar la problemática se requiere de un modelo robusto ante el ruido, que permita trabajar con datos no paramétricos, sea favorable a la distribución de los datos, pueda establecer un modelo mediante entrenamiento supervisado, con capacidad de ejecutar regresiones a partir de entrenamientos previos y permita realizar predicciones de valores de resistividad. Estas condiciones son cubiertas por tres modelos de ML,Random Forests (RF), Support Vector Machines (SVM) y Gradient Boosting Regression (GBR), e identificar mediante la comparación entre los modelos, la puntuación de predicción, la facilidad de implementación y ejecución en pruebas con datos reales, cual presenta el mejor rendimiento y ajuste con respecto a dotos reales, para su implementación en la adquisición geofísica. 
