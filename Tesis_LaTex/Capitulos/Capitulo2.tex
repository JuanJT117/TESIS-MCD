\chapter{DELIMITACIÓN Y PLANTEAMIENTO DEL PROBLEMA}
Durante un trabajo de prospección geofísica, al realizar adquisición de datos geoelectricos \textit{in situ}, no es posible conocer el resultado del trabajo hasta una vez realizado procesos de inversión de datos resultado de prospección, por lo que no se tiene certeza de si una lectura presenta inconsistencias o si forma parte de una respuesta esperada para determinado resultado.
Durante un trabajo de prospección geofísica, al realizar adquisición de datos geoeléctricos \textit{in situ}, no es posible conocer el resultado del trabajo hasta una vez realizado procesos de inversión de datos resultado de prospección, por lo que no se tiene certeza de si una lectura presenta inconsistencias o si forma parte de una respuesta esperada para determinado resultado.

El problema es no contar con un modelo que pueda anticipar el resultado de un modelo de inversión geológico, o al menos aproximar un resultado que pueda tener coherencia geológica en interpretación

Como primera aproximación a una solución se considerara un primer supuesto, en el cual se cuenta con respuestas de sondeos eléctricos verticales, más simple pero no menos complejo de interpretar, al contar con un solo segmento de señal, dentro de esta señal podemos encontrar distintas unidades geológicas, profundidad de acuífero, espesores de unidades y profundidad de exploración.

La propuesta de solución es mediante Aprendizaje automático (ML, por sus siglas en inglés), empleando una técnica de aprendizaje que permita identificar patrones que estén asociados a la respuesta de un modelo de inversión, de esta manera obtener un modelo pronóstico de la inversión, sirviendo de guía para incrementar la densidad de muestreo para mejorar el modelo y respuestas.  
