\chapter{DELIMITACIÓN Y PLANTEAMIENTO DEL PROBLEMA}

Durante un trabajo de prospección geoeléctrica, en adquisición de datos \textit{in situ}, no es posible la interpretación directa del trabajo hasta una vez realizado el procesado y modelado de datos, por lo que no se tiene certeza de si la adquisición realizada en campo representara con claridad el objeto de prospección, es decir, si el muestreo realizado logra alcanzar el objetivo de exploración, y por consiguiente, representar con claridad las unidades geológicas de un sitio en particular.

El muestreo planificado, siempre incorpora algún grado de ambigüedad, relacionado con a la calidad de información empleada en su preparación, en el, se infiere la distribución y espesores de las unidades geoeléctrica, a partir de estudios previo, reconocimiento geológico de las unidades presentes en el área o incluso exploraciones directas en el sitio de estudio, sin embargo por este método, incluso considerando exploración directa en la planeación, es imposible inferir las propiedades geoeléctricas y distribución real del subsuelo.  

Se evalúa como herramienta de pronostico y mejora de muestreo \textit{in situ} la implementación de técnicas de Machine Learning, a partir del entrenamiento de modelos de regresión, Random Forest (RF), Support Vector Machines (SVM) y Gradient Boosting Regression (GBR), con datos resultados  interpretados y calibrados por sondeo directo, de manera que esto permita identificar oportunidades de mejora en adquisición y muestro.

%Como primera aproximación a una solución se considerara un primer supuesto, en el cual se cuenta con respuestas de sondeos eléctricos verticales, más simple pero no menos complejo de interpretar, al contar con un solo segmento de señal, dentro de esta señal podemos encontrar distintas unidades geológicas, profundidad de acuífero, espesores de unidades y profundidad de exploración.\\

%La propuesta de solución es mediante Aprendizaje automático (ML, por sus siglas en inglés), empleando una técnica de aprendizaje que permita identificar errores en el patrón de muestreo, 
  
     