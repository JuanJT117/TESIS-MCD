\chapter{DELIMITACIÓN Y PLANTEAMIENTO DEL PROBLEMA}

Durante un trabajo de prospección geofísica, al realizar adquisición de datos geoeléctricos \textit{in situ}, no es posible conocer el resultado del trabajo hasta una vez realizado procesos de inversión de datos resultado de prospección, por lo que no se tiene certeza de si una lectura presenta inconsistencias o si forma parte de una respuesta esperada para determinado resultado.\\

La problemática radica en que el muestreo planeado, en la etapa de planeación de adquisición, puede no ser efectivo o no reflejar una distribución esperada, debido a que el medio que se prospecta carece de homogeneidad, siendo solo evidente durante la exploración directa del medio, lo que en muchos casos no se logra a identificar, sumando ambigüedad al proceso de interpretación.\\ 

Como primera aproximación a una solución se considerara un primer supuesto, en el cual se cuenta con respuestas de sondeos eléctricos verticales, más simple pero no menos complejo de interpretar, al contar con un solo segmento de señal, dentro de esta señal podemos encontrar distintas unidades geológicas, profundidad de acuífero, espesores de unidades y profundidad de exploración.\\

La propuesta de solución es mediante Aprendizaje automático (ML, por sus siglas en inglés), empleando una técnica de aprendizaje que permita identificar patrones que estén asociados a la respuesta de un modelo de inversión, de esta manera obtener un modelo pronóstico de la inversión, sirviendo de guía para incrementar la densidad de muestreo para mejorar el modelo y respuestas.
