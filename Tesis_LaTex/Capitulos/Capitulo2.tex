\chapter{DELIMITACIÓN Y PLANTEAMIENTO DEL PROBLEMA}

Durante un trabajo de prospección geoeléctrica, en al adquisición de datos \textit{in situ}, no es posible conocer el resultado del trabajo hasta una vez realizado el procesado de los mismos, por lo que no se tiene certeza de si la adquisición realizada en campo representara con claridad el objeto de prospección, es decir, si el muestreo realizado logra alcanzar el objetivo de exploración, y por consiguiente, representar con claridad las unidades geológicas de un sitio en particular.

El muestreo planificado para la etapa de adquisición, incorpora un grado alto de ambigüedad, debido a que no es posible conocer con certeza la distribución y espesores de las unidades geoeléctrica y por lo tanto no es factible un muestreo completamente efectivo, siendo solo parcialmente evidente durante la exploración directa del medio, ya que dicho procedimiento solo permite aprecia una porción ínfima del terreno.

Se plantea como herramienta de análisis y mejora de muestreo la implementación de técnicas de Machine Learning, empleando el aprendizaje mediante su entrenamiento con datos procesados y calibrados por sondeo directo, de manera que esto permita identificar oportunidades de mejora en el muestreo y la respuesta en general.

%Como primera aproximación a una solución se considerara un primer supuesto, en el cual se cuenta con respuestas de sondeos eléctricos verticales, más simple pero no menos complejo de interpretar, al contar con un solo segmento de señal, dentro de esta señal podemos encontrar distintas unidades geológicas, profundidad de acuífero, espesores de unidades y profundidad de exploración.\\

%La propuesta de solución es mediante Aprendizaje automático (ML, por sus siglas en inglés), empleando una técnica de aprendizaje que permita identificar errores en el patrón de muestreo, 
  
     